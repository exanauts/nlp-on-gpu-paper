\subsection{Null-space strategy}
\label{sec:kkt:nullspace}
The first strategy to solve the condensed KKT system~\eqref{eq:kkt:condensed}
is to reduce it to a dense matrix. This
method is known as a \emph{reduced Hessian method} in the
terminology of the optimization community \cite{biegler1995reduced}
and as a \emph{null-space strategy} in the linear algebra
community~\cite[Section 6]{benzi2005numerical}.

We suppose $G$ is full-rank.
Then there exists a permutation matrix $P \in \mathbb{R}^{n \times n}$
such that we can reorder the Jacobian of the equality constraint
$G \in \mathbb{R}^{m_e \times n}$ as $GP = \begin{bmatrix}
  G_d & G_u
\end{bmatrix}$, where $G_d$ is a $m_e \times m_e$ nonsingular matrix.
The matrix
\begin{equation}
  \label{eq:nullspace}
  Z = P \begin{bmatrix}
    - G_d^{-1} G_u \\ I
  \end{bmatrix} \; ,
\end{equation}
is a basis for the null-space of the Jacobian $G$, as $GZ = 0$.
In practice, the matrix $Z$ is never assembled explicitly.
If a LU decomposition of $G_d$ is available, then we can
implement $Z$ as a linear operator encoding the multiplications
$Z x$ and $Z^\top y$.

Interestingly, the null-space basis can be used to reduce the condensed
KKT system~\eqref{eq:kkt:condensed}.

\begin{proposition}[Reduced KKT system]
  Let $Z$ be the null-space basis defined in \eqref{eq:nullspace}.
  Let $\hat{p}_x$ be a particular solution satisfying $G \hat{p}_x = \hat{r}_2$.
  Let $(p_x, p_y)$ be
  solution of the condensed KKT system~\eqref{eq:kkt:condensed}.
  Then $p_x= \hat{p}_x + Z p_u$, with $p_u$ solution of the \emph{reduced KKT system}
  \begin{equation}
    \label{eq:kkt:reduced}
    \tag{$K_0$}
    Z^\top K Z p_u =
    Z^\top [\hat{r}_1 - K \hat{p}_x ] \; ,
  \end{equation}
  and $p_y$ is solution of the normal equation $G G^\top p_y = G (\hat{r}_1 - K p_x)$.
\end{proposition}
\begin{proof}
  A solution $(p_x, p_y)$ of \eqref{eq:kkt:condensed} satisfies
  \begin{equation}
    \label{eq:proof:reduced1}
    K p_x + G^\top p_y = \hat{r}_1 \; , \quad
    G p_x = \hat{r}_1  \; .
  \end{equation}
  Using the particular solution $\hat{p}_x$, we define
  $p_x = \hat{p}_x + Z p_u$. Replacing in \eqref{eq:proof:reduced1},
  we get $K (\hat{p}_x + Z p_u) = \hat{r}_1 - G^\top p_y$.
  Multiplying on the left by $Z^\top$ and using $Z^\top G^\top = 0$,
  we recover the expression in \eqref{eq:kkt:reduced}.
  The dual solution $p_y$ is solution of the overdetermined
  system of equations $G^\top p_y = \hat{r}_1 - K p_x$.
  Multiplying on the left by $G$, we get $G G^\top p_y =
  G (\hat{r}_1 - K p_x)$.
\end{proof}

The reduced matrix $Z^\top K Z$ appearing in the reduced KKT system
\eqref{eq:kkt:reduced} is usually dense, with dimension $(n-m_e) \times
(n - m_e)$. For that reason, the null-space strategy is effective
when the number of degrees of freedom $n - m_e$ is small.
The matrix $Z^\top K Z$ has a higher conditioning than the original
condensed matrix $K$, prohibiting a solution of \eqref{eq:kkt:reduced}
with iterative method~\cite{gould2001solution}. For that reason,
it is recommended to assemble $Z^\top K Z$ explicitly. Our previous
investigations have shown that if a LU factorization of $G_d$ is
available, we can assemble $Z^\top K Z$ efficiently on the GPU
using batched linear solves~\cite{pacaud2022condensed}.

A limitation of the null-space strategy is that
there exists no obvious permutation $P$ to build the null-space
matrix~\eqref{eq:nullspace} for generic nonlinear programs. However, for certain applications the
permutation is directly given by the structure of the problem
(coordinate decomposition).
Such problems usually arise in engineering, where one can split
the variable $x$ into two sets consisting of \emph{state variables} (dependent)
and \emph{control variables} (independent), respectively. Classical applications
are optimal control problems, optimal power flow (OPF) or PDE-constrained optimization.
An alternative is to compute an orthonormal decomposition using a QR factorization
to build the matrix $Z$, but such a strategy is not effective in the
large-scale regime.

The first method is based on a null-space
strategy and reduced the KKT system down to a dense matrix $Z^\top K Z$, easy
to factorize on the GPU. The downside is that assembling the reduced
matrix $Z^\top K Z$ can be challenging if the number of degrees of freedom
is too high.
