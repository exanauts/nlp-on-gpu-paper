\section{Solving KKT systems on the GPU}
The GPU has emerged as one of the new prominent landscape
for numerical computing. GPUs are using a SIMD formalism
that yields excellent throughput for parallelizing small-scale
operations, but their use remains limited as soon as the computational
pattern becomes more complex. With their heavy use of numerical
pivoting, sparse factorization algorithms are notoriously hard to implement
on a GPU for that exact reason. In particular, previous works have
shown that GPU-based linear solvers lag far behind their equivalent
on the CPU to factorize the system~\eqref{eq:kkt:augmented}
\cite{tasseff2019exploring,swirydowicz2021linear}. Two strategies
have emerged to overcome this challenge: densify the solution
of the KKT system~\cite{pacaud2022condensed} or use a sparse factorization that
does not require numerical pivoting~\cite{regev2023hykkt,shin2023accelerating}.
Interestingly, all methods are using the condensed KKT system~\eqref{eq:kkt:condensed}.

\subsection{Null-space strategy}
\label{sec:kkt:nullspace}
The first strategy to solve the condensed KKT system~\eqref{eq:kkt:condensed}
is to reduce it to a dense matrix. This
method is known as a \emph{reduced Hessian method} in the
terminology of the optimization community \cite{biegler1995reduced}
and as a \emph{null-space strategy} in the linear algebra
community~\cite[Section 6]{benzi2005numerical}.

We suppose $G$ is full-rank.
Then there exists a permutation matrix $P \in \mathbb{R}^{n \times n}$
such that we can reorder the Jacobian of the equality constraint
$G \in \mathbb{R}^{m_e \times n}$ as $GP = \begin{bmatrix}
  G_d & G_u
\end{bmatrix}$, where $G_d$ is a $m_e \times m_e$ nonsingular matrix.
The matrix
\begin{equation}
  \label{eq:nullspace}
  Z = P \begin{bmatrix}
    - G_d^{-1} G_u \\ I
  \end{bmatrix} \; ,
\end{equation}
is a basis for the null-space of the Jacobian $G$, as $GZ = 0$.
In practice, the matrix $Z$ is never assembled explicitly.
If a LU decomposition of $G_d$ is available, then we can
implement $Z$ as a linear operator encoding the multiplications
$Z x$ and $Z^\top y$.

Interestingly, the null-space basis can be used to reduce the condensed
KKT system~\eqref{eq:kkt:condensed}.

\begin{proposition}[Reduced KKT system]
  Let $Z$ be the null-space basis defined in \eqref{eq:nullspace}.
  Let $\hat{p}_x$ be a particular solution satisfying $G \hat{p}_x = \hat{r}_2$.
  Let $(p_x, p_y)$ be
  solution of the condensed KKT system~\eqref{eq:kkt:condensed}.
  Then $p_x= \hat{p}_x + Z p_u$, with $p_u$ solution of the \emph{reduced KKT system}
  \begin{equation}
    \label{eq:kkt:reduced}
    \tag{$K_0$}
    Z^\top K Z p_u =
    Z^\top [\hat{r}_1 - K \hat{p}_x ] \; ,
  \end{equation}
  and $p_y$ is solution of the normal equation $G G^\top p_y = G (\hat{r}_1 - K p_x)$.
\end{proposition}
\begin{proof}
  A solution $(p_x, p_y)$ of \eqref{eq:kkt:condensed} satisfies
  \begin{equation}
    \label{eq:proof:reduced1}
    K p_x + G^\top p_y = \hat{r}_1 \; , \quad
    G p_x = \hat{r}_1  \; .
  \end{equation}
  Using the particular solution $\hat{p}_x$, we define
  $p_x = \hat{p}_x + Z p_u$. Replacing in \eqref{eq:proof:reduced1},
  we get $K (\hat{p}_x + Z p_u) = \hat{r}_1 - G^\top p_y$.
  Multiplying on the left by $Z^\top$ and using $Z^\top G^\top = 0$,
  we recover the expression in \eqref{eq:kkt:reduced}.
  The dual solution $p_y$ is solution of the overdetermined
  system of equations $G^\top p_y = \hat{r}_1 - K p_x$.
  Multiplying on the left by $G$, we get $G G^\top p_y =
  G (\hat{r}_1 - K p_x)$.
\end{proof}

The reduced matrix $Z^\top K Z$ appearing in the reduced KKT system
\eqref{eq:kkt:reduced} is usually dense, with dimension $(n-m_e) \times
(n - m_e)$. For that reason, the null-space strategy is effective
when the number of degrees of freedom $n - m_e$ is small.
The matrix $Z^\top K Z$ has a higher conditioning than the original
condensed matrix $K$, prohibiting a solution of \eqref{eq:kkt:reduced}
with iterative method~\cite{gould2001solution}. For that reason
it is recommended to assemble $Z^\top K Z$ explicitly. Our previous
investigations have shown that if a LU factorization of $G_d$ is
available, we can assemble $Z^\top K Z$ efficiently on the GPU
using batched linear solves~\cite{pacaud2022condensed}.

A limitation of the null-space strategy is that
there exists no obvious permutation $P$ to build the null-space
matrix~\eqref{eq:nullspace} for generic nonlinear programs. However, for certain applications the
permutation is directly given by the structure of the problem
(coordinate decomposition).
Such problems usually arise in engineering, where one can split
the variable $x$ into two sets consisting resp. of \emph{state variables} (dependent)
and \emph{control variables} (independent). Classical applications
are optimal control problems, optimal power flow (OPF) or PDE-constrained optimization.
An alternative is to compute an orthonormal decomposition using a QR factorization
to build the matrix $Z$, but such strategy is not effective in the
large-scale regime.



\subsection{Golub \& Greif strategy}
\label{sec:kkt:golubgreif}
The Golub \& Greif~\cite{golub2003solving} strategy reformulates the KKT system
using an Augmented Lagrangian formulation.
It has been recently revisited in \cite{regev2023hykkt}
to solve the condensed KKT system~\eqref{eq:kkt:condensed} on the GPU.

The trick is to reformulate the condensed KKT system \eqref{eq:kkt:condensed} in the equivalent form
\begin{equation}
  \label{eq:kkt:hykkt}
  \begin{bmatrix}
    K_\gamma & G^\top \\
    G & 0
  \end{bmatrix}
  \begin{bmatrix}
    d_x \\ d_y
  \end{bmatrix}
  =
  \begin{bmatrix}
    \hat{r}_1 + \gamma G^\top \hat{r}_2 \\
    \hat{r}_2
  \end{bmatrix}
\end{equation}
where we have introduced the regularized matrix $K_\gamma := K + \gamma G^\top G$.
For a large-enough parameter $\gamma$, the matrix $K_\gamma$ is positive definite.
\begin{proposition}[\cite{regev2023hykkt}, Theorem 1, p.7]
  We suppose $G$ is full row rank. Then there exists $\gamma_{min}$
  such that, for all $\gamma > \gamma_{min}$, $K_\gamma$ is positive definite
  if and only if $Z^\top K Z$ is positive definite.
\end{proposition}

The linear solver HyKKT~\cite{regev2023hykkt}
exploits that property to solve \eqref{eq:kkt:hykkt} using a hybrid
direct-iterative method. The method solves the linear system
\eqref{eq:kkt:hykkt} using the following steps:
\begin{enumerate}
  \item Assemble $K_\gamma$ and factorize it using sparse Cholesky ;
  \item Solve the Schur complement of \eqref{eq:kkt:hykkt} using a conjugate gradient (CG)
    algorithm to recover the dual descent direction:
    \begin{equation}
      (G K_\gamma^{-1} G^\top) d_y = G K_\gamma^{-1} (\hat{r}_1 + \gamma G^\top \hat{r}_2) - \hat{r}_2 \; .
    \end{equation}
  \item Solve the system $K_\gamma d_x = \hat{r}_1 + \gamma G^\top \hat{r}_2 - G^\top d_y$
    to get the primal descent direction.
\end{enumerate}
The method uses a sparse Cholesky factorization used in conjunction with a CG algorithm.
The sparse Cholesky factorization has the advantage of being stable without
numerical pivoting, rendering the algorithm tractable on a GPU.
Each CG iteration requires the application of a backsolve with the
factors of $K_\gamma$, an operation that can be numerically demanding. For that reason, the
hybrid method HyKKT requires a small number of iterations to be numerically efficient.
Fortunately, the eigenvalues of the Schur-complement $G K_\gamma^{-1} G^\top$
converge to a single value as we increase the regularization parameter
$\gamma$ (\cite[Theorem 4]{regev2023hykkt}). The larger the $\gamma$, the less iterations is required in the CG algorithm.


\subsection{A novel equality relaxation strategy (SS)}


\subsection{Discussion}
We have introduced three different algorithms to solve
the KKT system on the GPU. On the contrary to classical CPU algorithm,
the three methods do not require computing a sparse LBL factorization of the KKT
system by using alternate reformulation based on the condensed KKT
system~\eqref{eq:kkt:condensed}.
The three methods have their advantages and their downsides.

The first method is based on a null-space
strategy and reduced the KKT system down to a dense matrix $Z^\top K Z$, easy
to factorize on the GPU. The downside is that assembling the reduced
matrix $Z^\top K Z$ can be challenging if the number of degrees of freedom
is too high. The second strategy uses an augmented Lagrangian formulation
of the KKT system. It requires a sparse Cholesky factorization (stable without numerical
pivoting, hence favorable for the GPU) and uses it to solve the Schur-complement system
using a CG algorithm. The performance of the method depends on a parameter $\gamma$, that
has to be tuned independently. The larger the $\gamma$, the faster is the convergence
in the CG algorithm, but the worst is the accuracy of the linear solve. Hence, a trade-off
has to be found. Eventually the third method uses an equality relaxation strategy to
solve the condensed KKT system~\eqref{eq:kkt:condensed} directly. It just requires
a sparse Cholesky solver to factorize~\eqref{eq:kkt:condensed}. However, the method
solves a relaxation of the original problem~\eqref{eq:problem}, limiting the accuracy
of the solution.

