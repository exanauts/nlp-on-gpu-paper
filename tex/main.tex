% Jinja Template for Latex file
% Copyright François Pacaud, 2023

\documentclass{article}

\usepackage{amsmath,amsfonts,amssymb,amsthm}
\usepackage{xcolor,bm,url}
\usepackage{booktabs}
\usepackage{tikz}

\newtheorem{theorem}{Theorem}[section]
\newtheorem{lemma}[theorem]{Lemma}
\newtheorem{corollary}[theorem]{Corollary}
\newtheorem{proposition}[theorem]{Proposition}
\theoremstyle{definition}
\newtheorem{definition}[theorem]{Definition}
\newtheorem{example}[theorem]{Example}
\theoremstyle{remark}
\newtheorem{remark}{Remark}

\title{Solving NLPs on GPU}
\author{TBD}
\date{\today}

\begin{document}
\maketitle

\tableofcontents

\section{Introduction}
Why solving NLP on GPUs?
\begin{itemize}
  \item SIMD abstraction and batch evaluation
  \item GPU's performance has increased significantly in the last decade, on the contrary to CPUs (post Moore law)
  \item Our recent work has proved that when used properly,
    GPUs can solve large-scale NLP 10x faster than state-of-the-art (quote PSCC paper)
\end{itemize}

\subsection{Current state-of-the-art (FP, SS)}

\paragraph{GPU for mathematical programming:}
General overview:
\begin{itemize}
  \item GPU used mostly in optimization for machine learning (Jaxopt, QPTH)
  \item Recent breakthrough in linear programming: a first-order
    algorithm --- PDHG --- beats Gurobi when running on the GPU.
\end{itemize}

\paragraph{GPU for nonlinear programming:}
\begin{itemize}
  \item Automatic differentiation: if AST represented with a SIMD abstraction,
    evaluation of the derivatives can be very fast (c.f. ExaModels)
  \item Linear solvers: current bottleneck, more than 90\% of the solution time spent
    inside the sparse linear solver.
    \begin{itemize}
      \item Previous generation of sparse solvers (e.g. SPRAL)
        was lagging behind compared to their CPU equivalents (HSL and
        Pardiso being state-of-the art for nonlinear programming).
      \item New solvers released by NVIDIA (hidden sparse Cholesky, and now cuDSS).
        Sparse Cholesky is getting competitive on the GPU, as it avoids
        extensive numerical pivoting.
      \item Alternative way to solve the KKT system:
        iterative methods (Krylov with $K_{2.5}$ formulation)
        and hybrid methods (Golub \& Greif, cf HyKKT)
    \end{itemize}
\end{itemize}

\subsection{Contributions (FP)}
\begin{enumerate}
  \item A new strategy based on equality relaxation
    and condensation of the KKT system
  \item Thorough comparison with alternative approaches
    (hybrid KKT solver and null-space solver), all implemented
    inside the same nonlinear optimization solver: MadNLP
  \item We test the latest sparse solver provided
    by NVIDIA: cuDSS
  \item Benchmark on large-scale OPF instances
    and others (to discuss)
\end{enumerate}


\section{Primal-dual interior-point method}
Basic recall on IPM.
\subsection{Problem's formulation and KKT conditions ( FP)}
We define the nonlinear program as
\begin{equation}
  \begin{aligned}
    \min_{x \in \mathbb{R}^n} \; & f(x) \\
    \text{subject to} \quad & g(x) = 0 \\
                            & h(x) \leq 0
  \end{aligned}
\end{equation}
TO DISCUSS: should we include bounds on $x$?

Using slack variables, the problem is reformulated as
\begin{equation}
  \begin{aligned}
    \min_{x \in \mathbb{R}^n, s \in \mathbb{R}^p} \; & f(x) \\
    \text{subject to} \quad & g(x) = 0 \, , \quad h(x) + s = 0  \\
                            & s \geq 0
  \end{aligned}
\end{equation}


\subsection{Solving the KKT conditions with the interior-point method (FP)}
Basic recall on IPM.

\paragraph{Augmented KKT system.}

\paragraph{Condensed KKT system.}



\section{Solving KKT systems on the GPU}
Goal: densification or avoid pivoting.
\subsection{Golub \& Greif strategy (HyKKT) (FP)}
\subsection{Null-space strategy (Argos) (FP)}
\subsection{A novel equality relaxation strategy (SS)}



\section{Numerical results}
TO DISCUSS: what do we want to show exactly? Some ideas:
\begin{itemize}
  \item Benchmark HyKKT, Argos and the equality relaxation strategy
    on OPF and SCOPF
    \begin{itemize}
      \item NB.: the null-space strategy requires identifying
        the degrees of freedom of the problem. I am afraid
        we have to stick to the OPF for the null-space strategy,
        unless we have another example in mind (optimal control?)
      \item Do a comparison CPU versus GPU
      \item Extend the benchmark beyond OPF? E.g. implement
        COPS benchmark in ExaModels
    \end{itemize}
\end{itemize}

\begin{itemize}
  \item SS: run a benchmark on moonshot
\end{itemize}


\section{Conclusion}
Future extensions:
\begin{itemize}
  \item Interplay with first-order method (PDHG for LP)
  \item Solving LP and QP problems on the GPU with IPM
  \item From Cholesky to LDL (regularization can
    ensure the KKT matrix is SQD, c.f. Orban, Saunders).
\end{itemize}

\bibliographystyle{apalike}
\bibliography{biblio.bib}

\end{document}
