% Jinja Template for Latex file
% Copyright François Pacaud, 2023

\documentclass{article}

\usepackage{amsmath,amsfonts,amssymb,amsthm}
\usepackage{xcolor,bm,url}
\usepackage{booktabs}
\usepackage{array}
\usepackage{tikz}

\newtheorem{theorem}{Theorem}[section]
\newtheorem{lemma}[theorem]{Lemma}
\newtheorem{corollary}[theorem]{Corollary}
\newtheorem{assumption}[theorem]{Assumption}
\newtheorem{proposition}[theorem]{Proposition}
\theoremstyle{definition}
\newtheorem{definition}[theorem]{Definition}
\newtheorem{example}[theorem]{Example}
\theoremstyle{remark}
\newtheorem{remark}{Remark}

\DeclareMathOperator{\diag}{diag}
\DeclareMathOperator{\argmin}{arg\,min}
\DeclareMathOperator{\nullspace}{null}
\DeclareMathOperator{\rangespace}{range}
\newcommand{\cond}{\kappa_2}
\newcommand{\epstol}{\mathbf{u}}
\newcommand{\cactive}{\mathcal{B}}
\newcommand{\cinactive}{\mathcal{N}}

% \title{Nonlinear programming on GPU: current state-of-the-art}
\title{Approaches to nonlinear programming on SIMD architectures}
\author{François Pacaud \and
Sungho Shin \and
Alexis Montoison \and
Michel Schanen \and
Mihai Anitescu
}
\date{\today}

\begin{document}
\maketitle

\begin{abstract}
  Solving nonlinear programs requires the
  successive solutions of large-scale indefinite symmetric matrices using
  sparse factorization routines that rely
  on notoriously hard-to-parallelize pivoting operations.
  A workaround is to reformulate the successive Karush-Kuhn-Tucker (KKT) systems
  in a form more amenable for the GPU,
  by reducing it down to a symmetric positive-definite condensed
  system whose factorization using Cholesky does not require pivoting.
  In this paper, we analyze the performance of two different condensed-space methods
  when used inside an interior-point solver.
  We prove that the condensed matrices exhibit structured ill-conditioning,
  limiting the loss of accuracy when computing a descent direction.
  We implement the two methods on current state-of-the-art SIMD architecture, using the newly
  released cuDSS solver on NVIDIA GPUs. We illustrate their strengths and weaknesses
  on large-scale instances taken from the PGLIB and the COPS benchmarks.
  Our findings indicate that current GPUs excel in rapidly computing results,
  achieving a moderate convergence tolerance---at a rate 10 times faster
  than that of state-of-the-art CPU routines---though they exhibit
  limited robustness. Such results are promising for the future of large-scale nonlinear programming on upcoming SIMD architectures.
\end{abstract}


% \tableofcontents


\section{Introduction}
Graphical processing units (GPUs) have become the most popular parallel computing architecture
for scientific computing, driven by their success in machine learning.
GPUs offer massive parallel computing capability for applications
that can exploit coarse-grain parallelism and high-memory bandwidth.
As such, the library CUDA has revolutionized high-performance computing (HPC) by democratizing parallel capabilities previously exclusive to expensive supercomputers.
In the post-Moore's law era, GPUs are becoming more power-efficient than CPUs for running parallel workloads, as they require fewer transistors to process multiple tasks in parallel.
This efficiency is further enhanced by technologies like ``single instruction, multiple data'' (SIMD), which allow GPUs to perform the same operation on multiple data simultaneously, maximizing throughput and efficiency.

While GPUs have made significant strides in enhancing machine learning applications, their adoption in the mathematical programming community has been relatively limited.
This limitation stems primarily from the fact that most optimization solvers were developed in the 1990s and are heavily optimized for CPU architectures.
Additionally, the utilization of GPUs has been impeded by the challenges associated with sparse matrix factorization routines, which are inherently difficult to parallelize on SIMD architectures. Nevertheless, recent years have witnessed notable advancements that are reshaping this landscape.

\begin{enumerate}
  \item \textbf{Improved sparse matrix operations}: The performance of sparse matrix operations has seen substantial improvements in the CUDA library, largely attributed to the integration of novel tensor cores in recent GPUs~\cite{markidis2018nvidia}.
  \item \textbf{Interest in batch optimization}: There is a growing interest in solving optimization problems in batch mode, for problems sharing the same structure but with different parameters~\cite{amos2017optnet,pineda2022theseus}.
  \item \textbf{Advancements in automatic differentiation}: GPUs offer unparalleled performance for automatic differentiation, benefiting both machine learning ~\cite{jax2018github} and scientific computing applications \cite{enzyme2021}. Engineering problems often exhibit recurring patterns throughout the model. Once these patterns are identified, they can be evaluated in parallel within a SIMD framework, enabling near speed-of-light performance~\cite{shin2023accelerating}.
  \item \textbf{Role in exascale computing}: With the emergence of new exascale architectures, GPUs have become indispensable in achieving high-performance computing, particularly in supercomputing environments.
\end{enumerate}

For all the reasons listed before, there is an increasing interest
to solve optimization problems on the GPU.

\subsection{Current state-of-the-art on GPU}

\paragraph{GPU for mathematical programming.}
The machine learning community has been a strong advocate for porting
mathematical optimization on the GPU. One of the most promising
applications is embedding mathematical programs inside neural networks,
a task that requires batching the solution of the optimization model
for the training algorithm to be
efficient~\cite{amos2017optnet,pineda2022theseus}.  This has led to
the development of prototype code solving thousands of (small)
optimization problems in parallel on the GPU.  However, it is not
trivial to adapt such code to solve large-scale optimization problems,
as the previous prototypes are reliant on dense linear solvers to
compute the descent direction.

For this reason, practitioners often resort to using first-order
methods on GPUs, leveraging level-1 and level-2 BLAS operations that
are more amenable to parallel computation.
First-order algorithms depend mostly on (sparse) matrix-vector operations, that run
very efficiently on modern GPUs. Hence, we can counterbalance
the relative inaccuracy of the first-order method by running more
iterations of the algorithm.
A recent breakthrough~\cite{lu2023cupdlp,lu2023cupdlp2} demonstrates
that a first-order algorithm can surpass the performance of Gurobi, a
commercial solver, in tackling large-scale linear programs. This
performance gain is made possible by executing the first-order
iterations solely on the GPU through an optimized codebase, thereby
solving the LP problems to $10^{-8}$ accuracy.


\paragraph{GPU for nonlinear programming.}
The success of first-order algorithms in classical mathematical programming
relies on the convexity of the problem. Thus, their approaches are nontrivial to replicate
in nonlinear programming. Most engineering problems embed complex
physical equations that are likely to break any convex structure in the problem.
Previous experiments on the OPF problem have shown that even a simple
algorithm as ADMM has trouble converging as soon as the convergence
tolerance is set below $10^{-3}$~\cite{kim2021leveraging}.

Thus, second-order methods continue to be a competitive option, particularly
for scenarios that require higher levels of accuracy and robust convergence.
Second-order algorithms require solving a Newton step at each
iteration, an operation relying on non-trivial sparse linear algebra operations.
The previous generation of GPU-accelerated sparse linear
solvers were lagging behind their CPU equivalents, as illustrated in
subsequent surveys~\cite{tasseff2019exploring,swirydowicz2021linear}.
Fortunately, sparse solvers on GPUs are getting increasingly better with the newest
generations of GPUs.
In particular, NVIDIA has released in November 2023
a new sparse direct solver that implements different sparse factorization routines: {\tt cuDSS}. Our
preliminary benchmark has shown {\tt cuDSS} is significantly
faster than the previous sparse solvers using NVIDIA GPUs.
The new NVIDIA Grace CPU architecture could also be a game changer in the future of sparse linear solvers, thanks to fast communication between the CPU and GPU.
Furthermore, variants of interior point methods have been proposed
that does not require the use of numerical pivoting.
As these algorithms do not require expensive numerical pivoting
operations, parallelized sparse solvers can be effectively exploited
within the solution algorithms.
Coupled with a GPU-accelerated automatic differentiation library and a
sparse Cholesky solver, these nonlinear programming solvers can solve
Optimal Power Flow (OPF) problems 10x faster than state-of-the-art
methods~\cite{shin2023accelerating}.

An alternative thread of research is investigating the solution of the Newton steps
on the GPU using iterative methods such as Krylov methods.
Iterative methods often require non-trivial reformulation of the Newton step to avoid
ill-conditioned matrices, which has limited their use inside interior-point
algorithms. New results are giving promising outlooks for convex problems~\cite{ghannad2022linear},
but nonconvex problems often require an Augmented Lagrangian reformulation
to be tractable~\cite{cao2016augmented,regev2023hykkt}. In particular,
\cite{regev2023hykkt} presents an interesting use of the Golub and Greif
hybrid method~\cite{golub2003solving} to solve the KKT systems arising in
the interior-point methods, with promising results on the GPU.
The null-space method, also known as the reduced Hessian strategy,
is also a good candidate for solving KKT systems on the GPU.
The null-space method reduces the KKT system down to
a medium-sized dense matrix, which then can be factorized efficiently on the GPU.
Our previous research has shown that the method plays nicely with the interior-point
methods if the number of degrees of freedom in the problem is relatively small~\cite{pacaud2022condensed}.

\subsection{Contributions}
In this article, we assess the current capabilities of modern GPUs
to solve large-scale nonconvex nonlinear programs to optimality.
We focus on the two condensed-space methods
introduced respectively in~\cite{regev2023hykkt,shin2023accelerating}.
We re-use classical results from~\cite{wright1998ill} to show
that for both methods, the condensed matrix exhibits
structured ill-conditioning that limits the loss of accuracy in
the descent direction (provided the interior-point algorithm satisfies
some standard assumptions).
We implement both algorithms inside the GPU-accelerated solver MadNLP,
and leverage the GPU-accelerated automatic differentiation
backend ExaModels~\cite{shin2023accelerating}.
The interior-point algorithm runs entirely on the GPU, from
the evaluation of the model (using ExaModels) to the solution of
the KKT system (using a condensed-space method running on the GPU).
We use CUDSS.jl \cite{Montoison_CUDSS}, a Julia interface to the NVIDIA library {\tt cuDSS} to solve
the condensed KKT systems.
We evaluate the strengths
and weaknesses of both methods, in terms of accuracy and runtime.
Extending beyond the classical OPF instances previously examined in our work, we incorporate large-scale problems sourced from the COPS nonlinear benchmark~\cite{dolan2004benchmarking}.
Our assessment involves comparing the performance achieved on the GPU with that of a state-of-the-art method executed on the CPU.
The findings reveal that the condensed-space methods demonstrate a remarkable 10x acceleration in solving large-scale OPF instances when utilizing the GPU.
However, performance outcomes on the COPS benchmark exhibit more variability.

\subsection{Notations}
By default, the norm $\|\cdot\|$ refers to the 2-norm.
We define the conditioning of a matrix $A$ as
$\cond(A) = \| A \| \|A^{-1} \|$.
For any real number $x$, we denote $\widehat{x}$ as its floating
point representation.
We denote $\epstol$ as the smallest positive number such that
$\widehat{x} \leq (1 + \tau) x$ for $|\tau| < \epstol$.
In double precision, $\epstol = 1.1 \times 10^{-16}$.
We use the following notations to proceed with our error analysis.
For $p \in \mathbb{N}$ and a positive variable $h$:
\begin{itemize}
  \item We write $x = O(h^p)$ if there exists a constant $b > 0$
    such that $\| x \| \leq b h^p$;
  \item We write $x = \Omega(h^p)$ if there exists a constant $a > 0$
    such that $\| x \| \geq a h^p$;
  \item We write $x = \Theta(h^p)$ if there exists two constants $0 < a < b$
    such that $a h^p \leq \| x \| \leq b h^p$.
\end{itemize}

\section{Primal-dual interior-point method}
The interior-point method (IPM) is among the most popular algorithms
to solve nonlinear programs. The basis of the algorithm is to
reformulate the Karush-Kuhn-Tucker (KKT) conditions of the nonlinear program as a smooth
system of nonlinear equations using a homotopy method~\cite{nocedal_numerical_2006}.
In a standard implementation, the
resulting system is solved iteratively with a Newton method (used in conjunction
with a line-search method for globalization). In this section, we
give a brief description of a nonlinear program in \S\ref{sec:ipm:problem}
and detail the Newton step solved at each IPM iteration in \S\ref{sec:ipm:kkt}.

\subsection{Problem's formulation and KKT conditions}
\label{sec:ipm:problem}
We are interested in solving the following nonlinear program:
\begin{equation}
  \label{eq:problem}
    \min_{x \in \mathbb{R}^n} \;  f(x)
\quad \text{subject to}\quad
\left\{
  \begin{aligned}
    & g(x) = 0 \; , ~ h(x) \leq 0 \; , \\
      & x \geq 0  \; ,
  \end{aligned}
\right.
\end{equation}
with $f:\mathbb{R}^n \to \mathbb{R}$ a real-valued function
encoding the objective, $g: \mathbb{R}^n \to \mathbb{R}^{m_e}$
encoding the equality constraints, and $h: \mathbb{R}^{n} \to
\mathbb{R}^{m_i}$ encoding the inequality constraints.
In addition, the variable $x$ is subject to simple bounds $x \geq 0$.
In what follows, we suppose that the functions $f, g, h$ are smooth
and twice differentiable.

We reformulate \eqref{eq:problem} using slack variables $s \geq 0$
into the equivalent formulation
\begin{equation}
  \label{eq:slack_problem}
    \min_{x \in \mathbb{R}^n, s \in \mathbb{R}^{m_i}} \;  f(x)
    \quad \text{subject to} \quad
    \left\{
  \begin{aligned}
    & g(x) = 0 \; , ~ h(x) + s = 0 \; , \\
      & x \geq 0  \; , ~ s \geq 0  \; .
  \end{aligned}
  \right.
\end{equation}
In~\eqref{eq:slack_problem}, the inequality constraints
are directly encoded inside the variable bounds.

We note by $y \in \mathbb{R}^{m_e}$ the multipliers associated
to the equality constraints and $z \in \mathbb{R}^{m_i}$ the multipliers
associated to the inequality constraints. Similarly, we note
by $(u, v) \in \mathbb{R}^{n + m_i}$ the multipliers associated
respectively to $x \geq 0$ and $s \geq 0$.
Using the multipliers $(y, z, v, w)$, we define the Lagrangian of \eqref{eq:slack_problem} as
\begin{equation}
  \label{eq:lagrangian}
  L(x, s; y, z, u, v) = f(x) + y^\top g(x) + z^\top \big(h(x) +s\big)
  - u^\top x - v^\top s \; .
\end{equation}
The KKT conditions of \eqref{eq:slack_problem} are:
\begin{subequations}
  \label{eq:kktconditions}
    \begin{align}
      & \nabla f(x) + \nabla g(x)^\top y + \nabla h(x)^\top z - u = 0 \\
      & z - v = 0 \\
      & g(x) = 0 \\
      & h(x) + s = 0 \\
      \label{eq:kktconditions:compx}
      & 0 \leq x \perp u \geq 0 \\
      \label{eq:kktconditions:comps}
      & 0 \leq s \perp v \geq 0
    \end{align}
\end{subequations}
The notation $x \perp u$ is a shorthand for the complementarity
condition $x_i u_i = 0$ (for all $i=1\cdots, n$).

The set of active constraints at a point $x$ is denoted by
\begin{equation}
  \mathcal{B}(x) := \{ i = 1, \cdots, m_i \; | \; h_i(x) = 0 \} \; .
\end{equation}
The inactive set is defined as the complement $\mathcal{N}(x) := \{1, \cdots, m_i \} \setminus \mathcal{B}(x)$.
We note $m_a$ the number of active constraints.
The active Jacobian is defined as $A(x) := \begin{bmatrix} \nabla g(x) \\ \nabla h_{\mathcal{B}}(x) \end{bmatrix} \in \mathbb{R}^{(m_e + m_a) \times n}$.

\begin{assumption}
  \label{hyp:ipm}
  Let $w^\star = (x^\star, s^\star, y^\star, z^\star, u^\star, v^\star)$ be a primal-dual solution
  satisfying the KKT conditions~\eqref{eq:kktconditions}. Let the following hold:
  \begin{itemize}
    \item Regularity: The Hessian $\nabla^2_{x x} L(\cdot)$ is Lipschitz continuous
      near $w^\star$;
    \item LICQ: the active Jacobian $A(x^\star)$ is full row-rank;
    \item SCS: for every $i \in \mathcal{B}(x^\star)$, $z_i^\star > 0$.
    \item SOSC: for every $v \in \nullspace\big(A(x^\star)\big)$,
      $v^\top \nabla_{x x}^2 L(w^\star) v > 0$.
  \end{itemize}
\end{assumption}


\subsection{Solving the KKT conditions with the interior-point method}
\label{sec:ipm:kkt}
The interior-point method aims at finding a stationary point
associated to the KKT conditions~\eqref{eq:kktconditions}.
The complementarity constraints \eqref{eq:kktconditions:compx}-\eqref{eq:kktconditions:comps}
render the KKT conditions non-smooth, complicating the solution of
the whole system~\eqref{eq:kktconditions}.
Instead, IPM uses a homotopy continuation method to solve a simplified
version of \eqref{eq:kktconditions}, parameterized by a barrier
parameter $\mu > 0$~\cite[Chapter 19]{nocedal_numerical_2006}.
For positive $(x, s, u, v) > 0$, we solve the system
$F_\mu(x, s, y, z, u, v) = 0$, with
\begin{equation}
  \label{eq:kkt_ipm}
  F_\mu(x, s, y, z, u, v) =
  \begin{bmatrix}
       \nabla f(x) + \nabla g(x)^\top y + \nabla h(x)^\top z - u  \\
       z - v  \\
       g(x)  \\
       h(x) + s  \\
       X u - \mu e  \\
       S v - \mu e
  \end{bmatrix}
   \; .
\end{equation}
We introduce in \eqref{eq:kkt_ipm} the diagonal matrices $X = \diag(x_1, \cdots, x_n)$
and $S = \diag(s_1, \cdots, s_{m_i})$.
As we drive the barrier parameter $\mu$ to $0$, we recover the original
KKT conditions~\eqref{eq:kktconditions}.

We note that at a fixed parameter $\mu$, the function $F_\mu(\cdot)$
is smooth. Hence, the system \eqref{eq:kkt_ipm} can be solved iteratively
using a regular Newton method. For a given primal-dual variable
$w_k := (x_k, s_k, y_k, z_k, u_k, v_k)$, the Newton step writes out
$w_{k+1} = w_k + \alpha_k d_k$, with $d_k$ a descent
direction being solution of the linear system
\begin{equation}
  \label{eq:newton_step}
  \nabla_w F(w_k) d_k = -F(w_k) \; .
\end{equation}
The step $\alpha_k$ is computed using a line-search algorithm, in a way
that ensures that the bounded variables remain positive
at the next primal-dual iterate (so that $(x_{k+1}, s_{k+1}, u_{k+1}, v_{k+1}) > 0$).
Once the iterates are sufficiently closed to the central path,
the IPM decreases the barrier term $\mu$ to find a solution closer to
the original KKT conditions~\eqref{eq:kktconditions}.

In IPM, the bulk of the workload is the computation of the Newton
step \eqref{eq:newton_step}, which involves (i) assembling the Jacobian
$\nabla_w F_\mu(w_k)$ and (ii) solving the linear system to compute
the descent direction $d_k$.
The Newton step~\eqref{eq:newton_step} expands as the $6 \times 6$
\emph{unreduced KKT system}:
\begin{equation}
  \label{eq:kkt:unreduced}
  \tag{$K_3$}
  \begin{bmatrix}
    W & 0 & G^\top & H^\top & -I & 0 \\
    0 & 0 & 0 & I & 0 & -I \\
    G & 0 & 0 & 0 & 0 & 0 \\
    H & I & 0 & 0 & 0 & 0 \\
    U & 0 & 0 & 0 & X & 0 \\
    0 & V & 0 & 0 & 0 & S
  \end{bmatrix}
  \begin{bmatrix}
    d_x \\
    d_s \\
    d_y \\
    d_z \\
    d_u \\
    d_v
  \end{bmatrix}
  % = -F_\mu(w_k) \; .
  = - \begin{bmatrix}
    \nabla_x L(w_k) \\
       % \nabla f(x_k) + \nabla g(x_k)^\top y_k + \nabla h(x_k)^\top z_k - v_k  \\
       z_k - v_k  \\
       g(x_k)  \\
       h(x_k) + s_k  \\
       X_k u_k - \mu e  \\
       S_k v_k - \mu e
  \end{bmatrix} \; ,
\end{equation}
where we have introduced the Hessian $W = \nabla^2_{x x} L(w_k)$ and
the two Jacobians $G = \nabla g(x_k)$, $H = \nabla h(x_k)$.
In addition, we define $U := \diag(u_1, \cdots, u_n)$
and $V = \diag(v_1, \cdots, v_{m_i})$.

\paragraph{Augmented KKT system.}
The system~\eqref{eq:kkt:unreduced} is not symmetric.
It is usual to remove the blocks associated
to the bound multipliers $(u, v)$ and solve instead the equivalent
$4 \times 4$ symmetric system, called the \emph{augmented KKT system}:
\begin{equation}
  \label{eq:kkt:augmented}
  \tag{$K_2$}
  \begin{bmatrix}
    W + D_x & 0 & G^\top & H^\top \\
    0 & D_s & 0& I \\
    G & 0 & 0 & 0 \\
    H & I & 0 & 0
  \end{bmatrix}
  \begin{bmatrix}
    d_x \\
    d_s \\
    d_y \\
    d_z
  \end{bmatrix}
  = - \begin{bmatrix}
    r_1 \\ r_2 \\ r_3 \\ r_4
       % \nabla f(x_k) + \nabla g(x_k)^\top y_k + \nabla h(x_k)^\top z_k   \\
       % z_k - w_k  \\
       % g(x_k)  \\
       % h(x_k) + s_k
  \end{bmatrix} \; ,
\end{equation}
with the diagonal matrices $D_x = X^{-1} U$ and $D_s = S^{-1} V$.
The right-hand-sides are given respectively by
$r_1 = \nabla f(x_k) + \nabla g(x_k)^\top y_k + \nabla h(x_k)^\top z_k + \mu X^{-1} e$,
$r_2 = z_k + \mu S^{-1} e$,
$r_3 = g(x_k)$,
$r_4 = h(x_k) + s_k$.
Once \eqref{eq:kkt:augmented} solved, we recover the
update on the bound multipliers with
$d_u = - X^{-1}(U d_x + X u_k - \mu e)$,
$d_v = - S^{-1}(V d_s + S v_k - \mu e)$.

The system \eqref{eq:kkt:augmented} is usually factorized using
an inertia-revealing LBL factorization.
Unfortunately, the block diagonal term
$D_x$ and $D_s$ are usually poorly conditioned, preventing
a solution with a Krylov iterative method.

\paragraph{Condensed KKT system.}
The $4 \times 4$ KKT system \eqref{eq:kkt:augmented} can be further
reduced down to a $2 \times 2$ system by eliminating the two blocks
$(d_s, d_z)$ associated to the inequality constraints.
The resulting system is called the \emph{condensed KKT system}:
\begin{equation}
  \label{eq:kkt:condensed}
  \tag{$K_1$}
  \begin{bmatrix}
    K & G^\top \\
    G & 0
  \end{bmatrix}
  \begin{bmatrix}
    d_x \\ d_y
  \end{bmatrix}
  =
  -
  \begin{bmatrix}
    r_1 + H^\top(D_s r_4 - r_2) \\ r_3
  \end{bmatrix}
  =:
  \begin{bmatrix}
    \bar{r}_1 \\ \bar{r}_2
  \end{bmatrix}
   \; ,
\end{equation}
where we have introduced the \emph{condensed matrix} $K := W + D_x + H^\top D_s H$.
Using the solution of the system~\eqref{eq:kkt:condensed},
we recover the updates on the slacks and inequality multipliers with
$d_s = -r_4 - Hd_x$ and $d_z = -r_2 - D_s d_s$.

\paragraph{Iterative refinement.}
Compared to \eqref{eq:kkt:unreduced},
the diagonal matrices $D_x$ and $D_s$ in \eqref{eq:kkt:augmented} introduce
an additional ill-conditioning in \eqref{eq:kkt:augmented}, amplified
in the condensed form~\eqref{eq:kkt:condensed}. For that reason, it
is recommended to refine the solution returned by the direct sparse linear
solver by using Richardson iterations on the original system~\eqref{eq:kkt:unreduced}
(see \cite[Section 3.10]{wachter2006implementation}).


\subsection{Discussion}
We have obtained three different formulations for the KKT system
appearing at each IPM iteration.
The original formulation \eqref{eq:kkt:unreduced} is not symmetric, but
has a better conditioning than the two alternatives \eqref{eq:kkt:augmented}
and \eqref{eq:kkt:condensed}.
The second formulation~\eqref{eq:kkt:augmented} is
used by default in state-of-the-art nonlinear solvers~\cite{wachter2006implementation,waltz2006interior}.
The system~\eqref{eq:kkt:augmented} is usually factorized using a LBL factorization: for sparse matrices, the Duff and Reid
multifrontal algorithm~\cite{duff1983multifrontal} is the favored method (as implemented in the linear solvers HSL
MA27 and MA57~\cite{duff2004ma57}).
On its end, the condensed KKT system~\eqref{eq:kkt:condensed} is often discarded,
as its conditioning is higher
than \eqref{eq:kkt:augmented} (implying less accurate solutions)
and the term $H^\top D_s H$ can lead to a dense condensed matrix if one column
of the Jacobian is dense. In addition,
the condensed matrix $K$ can potentially lead to additional fill-in
in the direct factorization~\cite[Section 19.3, p.571]{nocedal_numerical_2006}.
For that reason, Knitro~\cite{waltz2006interior} is the only solver
that supports computing the descent direction with \eqref{eq:kkt:condensed}.


%%% Local Variables:
%%% mode: latex
%%% TeX-master: "../main"
%%% End:

\section{Solving KKT systems on the GPU}
The GPU has emerged as the new prominent landscape for numerical computing.
GPUs employ a SIMD formalism that yields excellent throughput for parallelizing small-scale operations.
However, their utility remains limited when computational algorithms require global communication.
Sparse factorization algorithms, which heavily rely on numerical pivoting, pose significant challenges for implementation on GPUs due to this limitation. Previous research has demonstrated that GPU-based linear solvers significantly lag behind their CPU counterparts \cite{tasseff2019exploring,swirydowicz2021linear}.
One emerging strategy to address this challenge is to utilize sparse factorization techniques that do not necessitate numerical pivoting \cite{regev2023hykkt,shin2023accelerating},
by leveraging the structure of the condensed KKT system \eqref{eq:kkt:condensed}.

\subsection{Golub \& Greif strategy}
\label{sec:kkt:golubgreif}
The Golub \& Greif~\cite{golub2003solving} strategy reformulates the KKT system
using an Augmented Lagrangian formulation.
It has been recently revisited in \cite{regev2023hykkt}
to solve the condensed KKT system~\eqref{eq:kkt:condensed} on the GPU.
The trick is to reformulate the condensed KKT system \eqref{eq:kkt:condensed} in the equivalent form
\begin{equation}
  \label{eq:kkt:hykkt}
  \begin{bmatrix}
    K_\gamma & G^\top \\
    G & 0
  \end{bmatrix}
  \begin{bmatrix}
    d_x \\ d_y
  \end{bmatrix}
  =
  \begin{bmatrix}
    \bar{r}_1 + \gamma G^\top \bar{r}_2 \\
    \bar{r}_2
  \end{bmatrix} \; ,
\end{equation}
where we have introduced the regularized matrix $K_\gamma := K + \gamma G^\top G$.
If SOSC holds, the matrix $K_\gamma$ is positive definite for a large-enough parameter $\gamma$.
\begin{proposition}[\cite{regev2023hykkt}, Theorem 1, p.7]
  \label{prop:kkt:hykkt:pd}
  We suppose $G$ is full row rank. Then there exists $\gamma_{min}$
  such that, for all $\gamma > \gamma_{min}$, $K_\gamma$ is positive definite
  if and only if the reduced Hessian $Z^\top K Z$ is positive definite.
\end{proposition}
If $\gamma$ is large enough, we can prove that the conditioning
of $K_\gamma$ increases linearly with $\gamma$.
\begin{proposition}[\cite{regev2023hykkt}, Theorem 2, p.7]
  \label{prop:kkt:hykkt:cond}
  We suppose $G$ is full row rank. Then there exists $\gamma_{max}
  \geq \gamma_{min}$ such that for all $\gamma \geq \gamma_{max}$,
  $\cond(K_\gamma)$ increases linearly with $\gamma$.
\end{proposition}

The linear solver HyKKT~\cite{regev2023hykkt}
leverages the positive definiteness of $K_\gamma$ and solves
\eqref{eq:kkt:hykkt} using a hybrid direct-iterative method
that uses the following steps:
\begin{enumerate}
  \item Assemble $K_\gamma$ and factorize it using sparse Cholesky ;
  \item Solve the Schur complement of \eqref{eq:kkt:hykkt} using a conjugate gradient (CG)
    algorithm to recover the dual descent direction:
    \begin{equation}
      \label{eq:kkt:schurcomplhykkt}
      (G K_\gamma^{-1} G^\top) d_y = G K_\gamma^{-1} (\bar{r}_1 + \gamma G^\top \bar{r}_2) - \bar{r}_2 \; .
    \end{equation}
  \item Solve the system $K_\gamma d_x = \bar{r}_1 + \gamma G^\top \bar{r}_2 - G^\top d_y$
    to recover the primal descent direction.
\end{enumerate}
The method uses a sparse Cholesky factorization along with the conjugate gradient (CG) algorithm \cite{hestenes-stiefel-1952}.
The sparse Cholesky factorization has the advantage of being stable without
numerical pivoting, rendering the algorithm tractable on a GPU.
Each CG iteration requires the application of sparse triangular solves with the
factors of $K_\gamma$, operations that can be numerically demanding. For that reason,
HyKKT is efficient only if the CG solver converges in a small number of iterations.
Fortunately, the eigenvalues of the Schur-complement $S_\gamma := G K_\gamma^{-1} G^\top$
all converge to $\frac{1}{\gamma}$ as we increase the regularization parameter
$\gamma$ (\cite[Theorem 4]{regev2023hykkt}), meaning that $\lim_{\gamma \to \infty} \cond(S_\gamma) = 1$.
Because the convergence of the CG method depends on the number of distinct eigenvalues of $S_{\gamma}$, when $\gamma$ increases, the eigenvalues of $S_{\gamma}$ become more clustered, resulting in fewer iterations being required to solve \eqref{eq:kkt:schurcomplhykkt}.

\subsection{Lifted KKT System strategy}
\label{sec:kkt:sckkt}

We observe in \eqref{eq:kkt:condensed} that if all the constraints in the problem are inequalities, the system in \eqref{eq:kkt:condensed} becomes a $n \times n$ system which is guaranteed to be positive definite if the primal regularization parameter $\delta_x$ is adequately large. Furthermore, this parameter can be chosen dynamically using the inertia information of the system in \eqref{eq:kkt:condensed}. This motivates the Lifted KKT System strategy: we approximate the equalities with lifted inequalities.

In particular, we relax the equality constraints in \eqref{eq:problem} using a small relaxation parameter $\tau > 0$ (chosen based on the numerical tolerance of the optimization solver), and solve the relaxed problem
\begin{equation}
  \label{eq:problemrelaxation}
    \min_{x \in \mathbb{R}^n} \;  f(x)
\quad \text{subject to}\quad
     - \tau \leq g(x) \leq \tau \;,~  h(x) \leq 0  \; .
\end{equation}
The problem~\eqref{eq:problemrelaxation} has only inequality constraints. After introducing slack variables, the condensed KKT system
\eqref{eq:kkt:condensed} reduces to
\begin{equation}
  \label{eq:liftedkkt}
    K_k d_x = - r_1 - H_k^\top(D_s r_4 - r_2) \; .
\end{equation}
Using the standard inertia correction procedure, the parameter $\delta_x$ is set to a value high enough to ensure that  $K_k$ is positive definite. Therefore, $K_k$ can be factorized with Cholesky factorization, satisfying the key requirement for the implementation on the GPU. The relaxation causes error in the final solution, but the error is in the same order of the solver tolerance, and thus, do not significantly deteriorates the solution quality.

While this method can be implemented with small modification in the optimization solver, the presence of tight inequality in \eqref{eq:problemrelaxation} causes severe ill-conditioning throughout the IPM iterations. Thus,
choosing the solver tolerance appropriately is necessary to get a reliable convergence behavior.


\subsection{Discussion}
We have introduced two algorithms to solve
the KKT system on the GPU. On the contrary to classical CPU algorithms,
the two methods do not require computing a sparse \lblt factorization of the KKT
system by using alternate reformulation based on the condensed KKT
system~\eqref{eq:kkt:condensed}.

The first strategy uses an augmented Lagrangian formulation
of the KKT system. It requires a sparse Cholesky factorization (stable without numerical
pivoting, hence favorable for the GPU) and uses it to solve the Schur-complement system
using the CG algorithm. The performance of the method depends on a parameter $\gamma$, that
has to be tuned independently. The larger the $\gamma$, the faster is the convergence
in the CG algorithm, but the worst is the accuracy of the linear solve. Hence, a trade-off
has to be found. The second method uses an equality relaxation strategy to
solve the condensed KKT system~\eqref{eq:kkt:condensed} directly, parameterized
by a relaxation parameter $\tau$. It just requires
a sparse Cholesky solver to factorize~\eqref{eq:kkt:condensed}. However, the method
solves a relaxed problem, limiting the accuracy
of the solution.


%%% Local Variables:
%%% mode: latex
%%% TeX-master: "../main"
%%% End:

\section{Conditioning of the condensed KKT system}
The condensed matrix $K$ appearing in \eqref{eq:kkt:condensed} is known to be ill-conditioned close to a local optimum solution.
This behavior is amplified for the matrices $K_\gamma$ (for large $\gamma$) and $K_\tau$ (for small $\tau$).
For a generic linear system $Mx = b$, the relative error after perturbing the right-hand side by $\Delta b$ is bounded by
\begin{subequations}
  \label{eq:cond:defaultbond}
\begin{equation}
  \| \Delta x \| \leq \| M^{-1} \| \| \Delta b \| \;, \quad
  \frac{\| \Delta x \| }{\| x \|} \leq \cond(M) \, \frac{\| \Delta b \|}{\|b \|} \;.
\end{equation}
When the matrix is perturbed by $\Delta M$, the perturbed solution
$\widehat{x}$ satisfies $\Delta x = \widehat{x}- x =  - (M + \Delta M)^{-1} \Delta M \widehat{x}$.
If $\cond(M) \approx \cond(M + \Delta M)$, we have $M \Delta x \approx -\Delta M x$ (neglecting second-order terms),
giving the bounds
\begin{equation}
  \| \Delta x \| \leq \|M^{-1}\| \|\Delta M \| \|x \| \; , \quad
  \frac{\| \Delta x \|}{\|x\|} \leq \cond(M)\frac{\|\Delta M \|}{\|M\|} \; .
\end{equation}
\end{subequations}
The relative errors are bounded above by the conditioning $\cond(M)$,
indicating that without further assumptions the condensed matrix $K$ amplifies
the errors made throughout the algorithm.
Hence, it is legitimate to investigate the impact of the ill-conditioning
when solving the condensed system~\eqref{eq:kkt:condensed} with LiftedKKT or with HyKKT.
We will see that we can tighten the bounds in \eqref{eq:cond:defaultbond}
by exploiting the structured ill-conditioning of the condensed matrix $K$.
We base our analysis on \cite{wright1998ill}, where
the author has put a particular emphasis on the condensed KKT
system~\eqref{eq:kkt:condensed}, but without equality constraints. We generalize her results to the
matrix $K_\gamma$, which incorporates both equality and inequality
constraints. The results extend directly to $K_\tau$ (by setting $\gamma = 0$).

To alleviate the notations, we suppose that the primal variable
$x$ is unconstrained, leaving only the slack $s$ with bounded values in \eqref{eq:slack_problem}.
This is equivalent to include the bounds on the variable $x$ in the inequality constraints,
as $\bar{h}(x) \leq 0$ with $\bar{h}(x) := (h(x), -x)$.

\subsection{Centrality conditions}
We start the discussion by recalling important results porting
on the iterates of the interior-point algorithm~\cite{wright2001effects}.
Let $p = (x, s, y, z)$ the current primal-dual iterate (the multiplier $v$ is considered apart),
and $p^\star$ a solution of the KKT conditions~\eqref{eq:kktconditions}.
We denote $\delta(p, v) = \| (p, v) - (p^\star, v^\star) \|$ the Euclidean distance to the
primal-dual stationary point $p^\star$.
From \cite[Theorem 2.2]{wright2001effects}, if Assumption~\ref{hyp:ipm}
hold at $p^\star$ and $v^\star > 0$,
\begin{equation}
  \delta(p, v) = \Theta\left( \left\Vert \begin{bmatrix}
      \nabla_p L(p, v) \\ \min(v, s)
  \end{bmatrix}
  \right\Vert \right) \; .
\end{equation}
For $(s, v) > 0$, we define the \emph{duality measure} $\Xi(s, v)$ as the mapping
\begin{equation}
  \Xi(s, v) = s^\top v / m_i \; . % Sungho: m_i not defined here. 
\end{equation}
We suppose the iterates $(p, v)$ satisfy the \emph{centrality conditions}
\begin{subequations}
  \label{eq:centralitycond}
  \begin{align}
    & \| \nabla_p \mathcal{L}(p, v) \| \leq C \; \Xi(s, v) \;,  \\
    \label{eq:centralitycond:complement}
    & (s, v) > 0 \;,\quad s_i v_i \geq \alpha \, \Xi(s, v) \quad \forall i =1, \cdots, m_i \; , % Sungho: is this condition correct? Shouldn't s_iv_i be lower-bounded by a constant?
  \end{align}
\end{subequations}
for some constant $C > 0$ and $\alpha \in (0, 1)$.
Conditions~\eqref{eq:centralitycond:complement} ensures that the products
$s_i v_i$ are not too disparate in the diagonal term $D_s$.
This condition is satisfied in the solver Ipopt (see \cite[Equation (16)]{wachter2006implementation}).
We denote $\cactive = \cactive(x^\star)$ the active-set at the optimal solution $x^\star$,
and $\cinactive = \cinactive(x^\star)$ the inactive set.

\begin{proposition}[\cite{wright2001effects}, Lemma 3.2]
  \label{prop:cond:boundslack}
  Suppose $p^\star$ satisfies Assumption~\ref{hyp:ipm}.
  If the current primal-dual iterate $(p, v)$ satisfies the centrality
  conditions~\eqref{eq:centralitycond}, then
  \begin{subequations}
    \begin{align}
      i \in \mathcal{B} \implies s_i = \Theta(\Xi) \, , \quad v_i = \Theta(1) \;, \\
      i \in \mathcal{N} \implies s_i = \Theta(1) \, , \quad v_i = \Theta(\Xi) \; .
    \end{align}
  \end{subequations}
\end{proposition}
Using the centrality conditions \eqref{eq:centralitycond}, we can bound
the distance to the solution $\delta(p, v)$ with the duality measure $\Xi$.
\begin{theorem}[\cite{wright2001effects}, Theorem 3.3]
  Suppose $p^\star$ satisfies Assumption~\ref{hyp:ipm}.
  If the current primal-dual iterate $(p, v)$ satisfies the centrality
  conditions~\eqref{eq:centralitycond}, then
  \begin{equation}
    \delta(p, v) = O(\Xi) \; .
  \end{equation}
\end{theorem}

\subsection{Structured ill-conditioning}
We show that if the iterates $(p, v)$ satisfy
the centrality conditions~\eqref{eq:centralitycond}, then the
condensed matrix $K_\gamma$ has a structured ill-conditioning.

\subsubsection{Invariant subspaces in $K_\gamma$}
First, we follow the analysis presented in \cite{wright1998ill}
and show that the condensed matrix $K_\gamma$ can be decomposed as
\begin{equation}
  \label{eq:cond:svd}
  K_\gamma = \begin{bmatrix} U_L & U_S \end{bmatrix}
  \begin{bmatrix}
    \Sigma_L & 0 \\ 0 & \Sigma_S
  \end{bmatrix}
  \begin{bmatrix}
    U_L^\top \\ U_S^\top
  \end{bmatrix}
  \; ,
\end{equation}
with $U$ an orthogonal matrix and $\Sigma$ diagonal, where
the two diagonal matrices $\Sigma_L$ and $\Sigma_S$ are both better conditioned than $K_\gamma$.
The result is a consequence of the SVD decomposition,
provided that the singular values satisfy $\frac{\sigma_1}{\sigma_{p}} \leq \frac{\sigma_1}{\sigma_n}$
and $\frac{\sigma_{p+1}}{\sigma_{n}} \leq \frac{\sigma_1}{\sigma_n}$.

We recall that $K_\gamma = W + H^\top D_s H + \gamma G^\top G$, with
the diagonal matrix $D_s = S^{-1} V$.
In addition, we denote $m_a$ the cardinal of $\mathcal{B}$,
and $p := m_a + m_e$. We denote by
$H_{\cactive}$ the Jacobian of active inequality constraints, $H_{\cinactive}$ the
Jacobian of inactive inequality constraints and
$A := \begin{bmatrix} G^\top & H_{\cactive}^\top \end{bmatrix}^\top$ the active Jacobian.
We define the minimum and maximum active slack values as
\begin{equation}
  s_{min} = \min_{i \in \cactive} s_i \; , \quad
  s_{max} = \max_{i \in \cactive} s_i \; .
\end{equation}

To establish the decomposition of $K_\gamma$ as presented in \eqref{eq:cond:svd},
we modify the approach outlined in \cite[Theorem 3.2]{wright1998ill} to account for the additional
term $\gamma G^\top G$ arising from the equality constraints.
\begin{theorem}[Properties of $K_\gamma$]
  \label{thm:cond}
  Suppose the condensed matrix is evaluated at a primal-dual
  point $(p, \nu)$ satisfying~\eqref{eq:centralitycond},
  for sufficiently small $\Xi$.
  Let $\lambda_1, \cdots, \lambda_n$ be the $n$ eigenvalues of
  $K_\gamma$, ordered as $|\lambda_1| \geq  \cdots \geq |\lambda_n|$.
  Let $\begin{bmatrix} Y & Z \end{bmatrix}$ be an orthogonal
  matrix, where the columns of $Z$ span the null-space of
  $A$. Let $\underline{\sigma} =\min\{\frac{1}{\Xi}, \gamma\}$
  and $\overline{\sigma} = \max\{\frac{1}{s_{min}}, \gamma\}$.
  Then,
  \begin{enumerate}
    \item[(i)] The $p$ largest-magnitude eigenvalues of $K_\gamma$ are positive,
      with $\lambda_1 = \Theta(\overline{\sigma})$ and $\lambda_p = \Omega(\underline{\sigma})$.
    \item[(ii)] The $n-p$ smallest-magnitude eigenvalues of $K_\gamma$
      are $\Theta(1)$.
    \item[(iii)] If $0 < p < n$, then $\cond(K_\gamma) = \Theta(\overline{\sigma})$.
    \item[(iv)] There are orthonormal matrices $\widetilde{Y}$ and $\widetilde{Z}$ for
      simple invariant subspaces of $K_\gamma$, such that $Y - \widetilde{Y} = O(\underline{\sigma}^{-1})$
      and $Z - \widetilde{Z} = O(\underline{\sigma}^{-1})$.
  \end{enumerate}
\end{theorem}
\begin{proof}
  We start the proof by setting apart the inactive constraints from the active constraints in $K_\gamma$:
  \begin{equation}
    K_\gamma = W + H_{\cinactive}^\top S_{\cinactive}^{-1} V_{\cinactive} H_{\cinactive}
    + A^\top D_\gamma A \, ,
    \quad
    \text{with} \quad D_\gamma = \begin{bmatrix} S_{\cactive}^{-1} V_{\cactive} & 0 \\ 0 & \gamma I \end{bmatrix} \; .
  \end{equation}
  Using Assumption~\ref{hyp:ipm}, Lipschitz
  continuity implies that the Hessian and the inactive Jacobian
  are bounded: $W = O(1)$, $H_{\cinactive} = O(1)$.
  Proposition~\ref{prop:cond:boundslack} implies that
  $s_{\cinactive} = \Theta(1)$ and $v_{\cinactive} = \Theta(\Xi)$. We deduce:
  \begin{equation}
    \label{eq:cond:inactiveblock}
    H_{\cinactive}^\top S_{\cinactive}^{-1} V_{\cinactive} H_{\cinactive} = O(\Xi) \; .
  \end{equation}
  Hence, for small enough $\Xi$,
  the condensed matrix $K_\gamma$ is dominated by the block of active constraints:
  \begin{equation}
    K_\gamma = A^\top D_\gamma A + O(1) \; .
  \end{equation}
  Sufficiently close to the optimum $p^\star$, the constraints qualification
  in Assumption~\ref{hyp:ipm} implies that $A = \Theta(1)$ and has rank $p$.
  The eigenvalues $\{\eta_i\}_{i =1,\cdots,n}$ of $A^\top D_\gamma A$
  satisfy $\eta_i > 0$ for $i = 1,\cdots,p$ and $\eta_i = 0$ for $i = p+1, \cdots, n$.
  As $s_{\cactive} = \Theta(\Xi)$ and $v_{\cactive} = \Theta(1)$
  (Proposition~\ref{prop:cond:boundslack}), the smallest diagonal
  element in $D_\gamma$ is $\Omega(\min\{\frac{1}{\Xi}, \gamma\})$
  and the largest diagonal element is $\Theta(\max\{\frac{1}{s_{min}}, \gamma\})$.
  Hence,
  \begin{equation}
    \eta_1 = \Theta(\overline{\sigma}) \; , \quad
    \eta_p = \Omega(\underline{\sigma}) \; .
  \end{equation}
  Using \cite[Lemma 3.1]{wright1998ill}, we deduce $\lambda_1 = \Theta(\overline{\sigma})$
  and $\lambda_p = \Omega(\underline{\sigma})$, proving the first result (i).

  Let $L_\gamma = A^\top D_\gamma A$.
  We have that
  \begin{equation}
    \begin{bmatrix}
      Z^\top \\ Y^\top
    \end{bmatrix}
    L_\gamma \begin{bmatrix}Z & Y \end{bmatrix}
    = \begin{bmatrix}
      L_1 & 0 \\
      0 & L_2
    \end{bmatrix} \; ,
  \end{equation}
  with $L_1 = 0$ and $L_2 = Y^\top L_\gamma Y$.
  The smallest eigenvalue of $L_2$ is $\Omega(\underline{\sigma})$
  and the matrix $E := K_\gamma - L_\gamma$ is $O(1)$.
  By applying \cite[Theorem 3.1, (ii)]{wright1998ill},
  the $n - p$ smallest eigenvalues in $K_\gamma$ differ by
  $\Omega(\underline{\sigma}^{-1})$ from those of the reduced Hessian $Z^\top K_\gamma Z$.
  In addition, \eqref{eq:cond:inactiveblock} implies
  that $Z^\top K_\gamma Z - Z^\top W Z = O(\Xi)$. Using SOSC,
  $Z^\top W Z$ is positive definite for small enough $\Xi$, implying
  all its eigenvalues are $\Theta(1)$. Using again \cite[Lemma 3.1]{wright1998ill},
  we get that the $n-p$ smallest eigenvalues in $K_\gamma$ are $\Theta(1)$,
  proving (ii). The third point (iii) is proved by combining
  (i) and (ii) (provided $0 < p < n$).
  Eventually, point (iv) is proven by using \cite[Theorem 3.1 (i)]{wright1998ill}.
\end{proof}

\begin{corollary}
  The condensed matrix $K_\gamma$ can be decomposed as
  \eqref{eq:cond:svd}, with $\Sigma_L \in \mathbb{R}^{p \times p}$ and $\Sigma_S \in \mathbb{R}^{(n-p) \times (n-p)}$.
  In addition, for suitably chosen $Y$ and $Z$,
  \begin{equation}
    \label{eq:cond:invariantsubpsace}
    U_L - Y = O(\underline{\sigma}^{-1}) \; , \quad
    U_S - Z = O(\underline{\sigma}^{-1}) \; .
  \end{equation}
\end{corollary}
\begin{proof}
  According to Theorem~\ref{thm:cond}, the $p$ largest eigenvalues of $K_\gamma$ are large
  and well separated from the $n - p$ smallest eigenvalues. Using
  the spectral theorem, we obtain the decomposition as \eqref{eq:cond:svd}.
  Using Theorem \ref{thm:cond}, part (iv), we obtain the result
  in \eqref{eq:cond:invariantsubpsace}.
\end{proof}
Theorem~\ref{thm:cond} gives us a deeper insight into the structure
of the condensed matrix $K_\gamma$.
Using equation~\eqref{eq:cond:invariantsubpsace}, we observe
we can assimilate the large-space of $K_\gamma$ with $\rangespace(A^\top)$
and the small space with $\nullspace(A)$.
The decomposition~\eqref{eq:cond:svd} leads to the following relations
\begin{equation}
  \label{eq:cond:boundinvariantsubspace}
  \begin{aligned}
    & \| K_\gamma \| = \| \Sigma_L \| = \Theta(\overline{\sigma}) \; , &
    \Sigma_L^{-1} = O(\underline{\sigma}^{-1})  \;, \\
    & \| K_\gamma^{-1} \| = \| \Sigma_S^{-1} \| = \Theta(1) \, , &
  \Sigma_S = \Theta(1) \, .
  \end{aligned}
\end{equation}
The condition of $\Sigma_L$ depends on $\cond(A)$, $\cond(V)$
and the ratio $\frac{s_{max}}{s_{min}} = O(\Xi \overline{\sigma})$.
The condition of $\Sigma_S$ reflects the condition of the reduced Hessian $Z^\top W Z$.

\begin{remark}
  Theorem~\ref{thm:cond} (iii) tells us that $\cond(K_\gamma) = \Theta(\overline{\sigma})$,
  meaning that if $\gamma$ is large-enough to satisfy $\gamma \geq \frac{1}{s_{min}}$, then
  the conditioning $\cond(K_\gamma)$ increases linearly with $\gamma$. In
  other words, the value $\frac{1}{s_{min}}$ gives us a lower-bound for $\gamma_{max}$
  in Proposition~\ref{prop:kkt:hykkt:cond}.
\end{remark}

\subsubsection{Numerical accuracy of the condensed matrix $K_\gamma$}
We are now interested in bounding the error made in
the computation of the condensed matrix $K_\gamma$ in floating
point arithmetic.

In direct contrast with \cite[Section 4.1]{wright1998ill}, the slack variable
$s$ appearing in the diagonal matrix $S^{-1} V$ is not subject to cancellation:
the interior-point algorithm is passing exactly the value of the slack
variables to the linear solver (this would not be the case if the slack
$s_i$ was replaced by the nonlinear function $h_i(x)$). Hence,
the error arising from the multiplication by $S^{-1}$ is only $O(\epstol)$,
which is significantly better than the error in $O(\epstol / s_{min})$ we would obtain
in the cancellation case.

In floating-point arithmetic, the condensed matrix $K_\gamma$ is evaluated as
\begin{multline*}
  \widehat{K}_\gamma = W + E + (A + \Delta A)^\top (D_\gamma + \Delta D_\gamma) (A + \Delta A) \\
  + (H_{\cinactive} + \Delta H_{\cinactive})^\top S_{\cinactive}^{-1} V_{\cinactive} (H_{\cinactive} + \Delta H_{\cinactive}) \; ,
\end{multline*}
with $\| \Delta H_{\cinactive} \| = O(\epstol \| H_{\cinactive}\|)$, $\| \Delta A \| = O(\epstol \| A\|)$
and $E$ symmetric with $\|E \| = O(\epstol \overline{\sigma})$.
We have
\begin{equation}
A^\top D_\gamma A = \Theta(\overline{\sigma}) \; , \quad
A^\top \Delta D_\gamma A = O(\epstol \overline{\sigma}) \; .
\end{equation}
We deduce that the perturbation $\Delta K_\gamma$ is bounded by $O(\epstol \overline{\sigma})$.
\begin{proposition}
  \label{prop:cond:boundcondensedmatrix}
  In floating-point arithmetic, the perturbation
  of the condensed matrix $\Delta K_\gamma$ satisfies
  $\Delta K_\gamma := \widehat{K_\gamma} - K_\gamma  = O(\epstol \overline{\sigma})$.
\end{proposition}
The perturbation $\Delta K_\gamma$ displays no specific structure.
Hence, we should determine under which
conditions the perturbed matrix $\widehat{K}_\gamma$
keeps the structure highlighted in \eqref{eq:cond:svd}.
We know that the smallest eigenvalue $\eta_p$ of $A^\top D_\gamma A$
is $\Omega(\underline{\sigma})$. As mentioned in
\cite[Section 3.4.2]{wright1998ill}, the perturbed matrix
$\widehat{K}_\gamma$ has $p$ large eigenvalues
bounded below by $\underline{\sigma}$ if the perturbation
is much smaller than the eigenvalue $\eta_p$:
\begin{equation}
  \label{eq:cond:perturbationbound}
  \| \Delta K_\gamma \| \ll \eta_p  \; .
\end{equation}
However, the bound in Proposition~\ref{prop:cond:boundcondensedmatrix} is too loose
for \eqref{eq:cond:perturbationbound} to hold without any further assumption.
As $\eta_p = \Omega(\underline{\sigma})$, we deduce $\frac{\|K_\gamma \|}{\eta_p}
= O(\overline{\sigma}\underline{\sigma}^{-1})$.
If we suppose in addition the ratio $\overline{\sigma}\underline{\sigma}^{-1}$ is close to $1$,
then $\|\Delta K_\gamma\| = O(\epstol \overline{\sigma})$ can be replaced by
$\| \Delta K_\gamma\|= O(\epstol \underline{\sigma})$ which in turns implies \eqref{eq:cond:perturbationbound}.

\subsubsection{Numerical solution of the condensed system}
We are interested in estimating the relative error
made when solving the system $K_\gamma x = b$ in floating
point arithmetic. We suppose $K_\gamma$ is factorized using
a backward-stable Cholesky decomposition. The computed
solution $\widehat{x}$ is solution of a perturbed system
$\widetilde{K}_\gamma \widehat{x} = b$, with $\widetilde{K}_\gamma
= K_\gamma + \Delta_s K_\gamma$ and $\Delta_s K_\gamma$ a symmetric matrix satisfying,
\begin{equation}
  \label{eq:cond:backwardstable}
  \|\Delta_s K_\gamma\| \leq \epstol \varepsilon_n \|K_\gamma\| \;,
\end{equation}
for $\epsilon_n$ a small constant depending on the dimension $n$.
We need the following additional assumptions to
ensure (a) the Cholesky factorization runs to completion
and (b) we can incorporate the backward-stable perturbation $\Delta_s K_\gamma$
in the generic perturbation $\Delta K_\gamma$ introduced in
Proposition~\ref{prop:cond:boundcondensedmatrix}.
\begin{assumption} Let $(p, v)$ be the current primal-dual iterate. We assume:
  \begin{itemize}
    \item[(a)] $(p, v)$ satisfies the centrality conditions~\eqref{eq:centralitycond}.
    \item[(c)] The parameter $\gamma$ satisfies $\gamma = \Theta(\Xi^{-1})$.
    \item[(c)] The duality measure is large enough relative to the precision $\epstol$: $\epstol \ll \Xi$.
    \item[(d)] The primal step $\widehat{x}$ is computed using a backward
      stable method satisfying \eqref{eq:cond:backwardstable} for a small constant
      $\varepsilon_n$.
  \end{itemize}
  \label{hyp:cond:wellcond}
\end{assumption}
Condition (a) implies that
$s_{min} = \Theta(\Xi)$ and $s_{max} = \Theta(\Xi)$ (Proposition \ref{prop:cond:boundslack}).
Condition (b) supposes in addition $\gamma = \Theta(\Xi^{-1})$, making
the matrix $\Sigma_L$ well-conditioned with
$\underline{\sigma} = \Theta(\Xi^{-1})$,
$\overline{\sigma} = \Theta(\Xi^{-1})$ and $\overline{\sigma}/\underline{\sigma} = \Theta(1)$.
Condition (c) ensures that the conditioning of $\cond(K_\gamma) = \Theta(\overline{\sigma})$
satisfies $\cond(K_\gamma) \epstol \ll 1$,
ensuring the Cholesky factorization runs to completion.
Condition (d) tells us that the perturbation caused by the Cholesky
factorization is $\Delta_s K_\gamma = O(\epstol \| K_\gamma\|)$. As
\eqref{eq:cond:boundinvariantsubspace} implies $\|K_\gamma \| = \Theta(\Xi^{-1})$,
we can incorporate $\Delta_s K_\gamma$ in the perturbation
$\Delta K_\gamma$ given in Proposition~\ref{prop:cond:boundcondensedmatrix}.

We denote $x$ the solution of the linear system $K_\gamma x = b$
in exact arithmetic, and $\widehat{x}$ the solution of
the perturbed system $\widehat{K}_\gamma \widehat{x} = \widehat{b}$
in floating-point arithmetic. We are interested in bounding
the error $\Delta x = \widehat{x} - x$. We
recall that every vector $x \in \mathbb{R}^n$ decomposes as
\begin{equation}
  x = U_L x_L + U_S x_S = Y x_Y + Z x_Z \; .
\end{equation}

\paragraph{Impact of right-hand-side perturbation.}
Using \eqref{eq:cond:svd}, the inverse of
$K_\gamma$ satisfies
\begin{equation}
  \label{eq:cond:inversecondensed}
  K_\gamma^{-1}  = \begin{bmatrix} U_L & U_S \end{bmatrix}
  \begin{bmatrix}
    \Sigma_L^{-1} & 0 \\ 0 & \Sigma_S^{-1}
  \end{bmatrix}
  \begin{bmatrix}
    U_L^\top \\ U_S^\top
  \end{bmatrix}
  \; .
\end{equation}
Hence, if we solve the system for $\widehat{b} := b + \Delta b$,
$\Delta x = K_\gamma^{-1} \Delta b$ decomposes as
\begin{equation}
  \begin{bmatrix}
    \Delta x_L \\ \Delta x_S
  \end{bmatrix}
  =
  \begin{bmatrix}
    \Sigma_L^{-1} & 0 \\ 0 & \Sigma_S^{-1}
  \end{bmatrix}
  \begin{bmatrix}
    \Delta b_L \\ \Delta b_S
  \end{bmatrix}
  \; ,
\end{equation}
which in turn implies the following bounds:
\begin{equation}
  \label{eq:cond:rhserror}
     \| \Delta x_L \| \leq \| \Sigma_L^{-1} \| \| \Delta b_L \| \; ,\quad
    \| \Delta x_S \| \leq \| \Sigma_S^{-1} \| \| \Delta b_S \| \; .
\end{equation}
As $\Sigma_L^{-1} = O(\Xi)$ and $\Sigma_S^{-1} = \Theta(1)$,
we deduce that the error $\Delta x_L$ is smaller by a factor
of $\Xi$ than the error $\Delta x_S$. The total error
$\Delta x = U_L \Delta x_L + U_S \Delta x_S$ is bounded by
\begin{equation}
  \label{eq:cond:rhserrorfull}
  \| \Delta x \|
  \leq  \| \Sigma_L^{-1} \| \| \Delta b_L \| + \| \Sigma_S^{-1} \| \| \Delta b_S \| =
  O(\|\Delta b \|) \; .
\end{equation}

\paragraph{Impact of matrix perturbation.}
As $\|\Delta K_\gamma\| \ll \|K_\gamma\|$, we have that
\begin{equation}
  \label{eq:cond:invperturbed}
  \begin{aligned}
    (K_\gamma + \Delta K_\gamma)^{-1} &= (I + K_\gamma^{-1} \Delta K_\gamma)^{-1} K_\gamma^{-1} \\
                                      &= K_\gamma^{-1} - K_\gamma^{-1}\Delta K_\gamma K_\gamma^{-1} + O(\|\Delta K_\gamma\|^2) \; .
  \end{aligned}
\end{equation}
Using \eqref{eq:cond:inversecondensed} and noting $\Delta K_\gamma = \begin{bmatrix}
  \Gamma_L \\ \Gamma _S
\end{bmatrix}$, the first-order error is given by
\begin{equation}
  \label{eq:cond:inversecondensederror}
  K_\gamma^{-1}\Delta K_\gamma K_\gamma^{-1}  =
U_L \Sigma_L^{-1} \Gamma_L \Sigma_L^{-1}U_L^\top  +
  U_S \Sigma_S^{-1} \Gamma_S \Sigma_S^{-1}U_S^\top  \;.
\end{equation}
Using \eqref{eq:cond:perturbationbound} and $(\Gamma_L, \Gamma_S)= O( \Xi^{-1}\epstol)$,
we deduce that the error made in the large-space is $O(\Xi\epstol)$ whereas
the error in the small-space is $O(\Xi^{-1}\epstol )$.

\subsection{Solution of the condensed KKT system}
We use the relations~\eqref{eq:cond:rhserror} and \eqref{eq:cond:inversecondensederror}
to bound the error made when solving the condensed KKT system~\eqref{eq:kkt:condensed}
in floating-point arithmetic.
In all this section, we assume that
the primal-dual iterate $(p,v)$ satisfies Assumption~\ref{hyp:cond:wellcond}.
Using \cite[Corollary 3.3]{wright2001effects}, the solution $(d_x, d_y)$ of the
condensed KKT system \eqref{eq:kkt:condensed} in exact arithmetic satisfies
$(d_x, d_y) = O(\Xi)$.
In \eqref{eq:kkt:condensed}, the RHS $\bar{r}_1$ and $\bar{r}_2$
evaluate in floating-point arithmetic as
\begin{equation}
  \label{eq:cond:condensedrhs}
  \left\{
  \begin{aligned}
    \bar{r}_1 &= - \widehat{r}_1 + \widehat{H}^\top\big(\widehat{D}_{s} \widehat{r}_{4} - \widehat{r}_{2} \big) \;, \\
     \bar{r}_2 &= -\widehat{r}_3 \; .
  \end{aligned}
  \right.
\end{equation}
Using basic floating-point arithmetic, we get
$\widehat{r}_1 = r_1 + O(\epstol)$,
$\widehat{r}_3 = r_3 + O(\epstol)$,
$\widehat{r}_4 = r_4 + O(\epstol)$.
The error in the right-hand-side $r_2$ is impacted by the term $\mu S^{-1}e$:
under Assumption~\ref{hyp:cond:wellcond}, it impacts differently
the active and inactive components:
$\widehat{r}_{2,\cactive}= r_{2,\cactive} + O(\epstol)$ and
$\widehat{r}_{2,\cinactive}= r_{2,\cinactive} + O(\Xi \epstol)$.
Similarly, the diagonal matrix $\widehat{D}_s$ retains full accuracy only
w.r.t. the inactive components: $\widehat{D}_{s,\cactive} = D_{s,\cactive} + O(\Xi^{-1} \epstol)$
and $\widehat{D}_{s,\cinactive} = D_{s,\cinactive} + O(\Xi \epstol)$.

\subsubsection{Solution with HyKKT}
We analyze the accuracy achieved when we solve the condensed system~\eqref{eq:kkt:condensed}
using HyKKT,
and show that the error remains reasonable even for large values of
the regularization parameter $\gamma$.

\paragraph{Initial right-hand-side.}
In floating-point arithmetic, the initial right-hand side in \eqref{eq:kkt:schurcomplhykkt}
is evaluated as
$\widehat{r}_\gamma :=\widehat{G} \widehat{K}_\gamma^{-1} (\bar{r}_1 + \gamma \widehat{G}^\top \bar{r}_2) - \bar{r}_2$. Using \eqref{eq:cond:condensedrhs}, we have
\begin{equation}
  \label{eq:cond:boundderivationhykkt}
  \begin{aligned}
  \bar{r}_1 + \gamma \widehat{G}^\top \bar{r}_2 &=
- \widehat{r}_1 + \gamma \widehat{G}^\top \widehat{r}_3+ \widehat{H}^\top\big(\widehat{D}_{s} \widehat{r}_{4} - \widehat{r}_{2} \big) \\
&=  -
\underbrace{\widehat{r}_1}_{O(\epstol)} +
\underbrace{
\widehat{H}_{\cinactive}^\top\big(\widehat{D}_{s,\cinactive} \widehat{r}_{4,\cinactive} - \widehat{r}_{2,\cinactive} \big)}_{O(\Xi \epstol)}
+ \underbrace{\widehat{A}^\top \begin{bmatrix}
  \widehat{D}_{s,\cactive} \widehat{r}_{4,\cactive} - \widehat{r}_{2,\cactive}  \\
  \gamma \widehat{r}_3
\end{bmatrix}}_{O(\Xi^{-1}\epstol)} \; .
  \end{aligned}
\end{equation}
We denote $\widehat{s}_\gamma = \bar{r}_1 + \gamma \widehat{G}^\top \bar{r}_2 =
Y \widehat{s}_Y + Z \widehat{s}_Z$.
The error decomposes as $\Delta s_\gamma = Y \Delta s_Y Y + Z \Delta s_Z
= U_L \Delta s_L + U_S \Delta s_S$.
Using \eqref{eq:cond:boundderivationhykkt},
we have $\Delta s_Y = O(\Xi^{-1} \epstol)$ and $\Delta s_Z = O(\epstol)$.
Using \eqref{eq:cond:invariantsubpsace}, we deduce
$\Delta s_L = U_L^\top \Delta s_\gamma = O(\Xi^{-1} \epstol)$ and
$\Delta s_S = U_S^\top \Delta s_\gamma = O(\epstol)$.
We obtain using \eqref{eq:cond:boundinvariantsubspace} and \eqref{eq:cond:inversecondensed}
that the error in the large space $\Delta s_L$ annihilates in the backsolve:
\begin{equation}
  \label{eq:cond:boundhykkt1}
  K_\gamma^{-1} \Delta s_\gamma = U_L \Sigma_L^{-1} \Delta s_L + U_S \Sigma_S^{-1} \Delta s_S  = O(\epstol)
  \; .
\end{equation}
Using \eqref{eq:cond:invperturbed}, we get
\begin{equation}
  \widehat{G} \widehat{K}_\gamma^{-1} \Delta s_\gamma \approx
  \widehat{G} (I - K_\gamma^{-1}\Delta K_\gamma) K_\gamma^{-1} \Delta s_\gamma \; .
\end{equation}
Using \eqref{eq:cond:boundhykkt1}, the first term is $\widehat{G} K_\gamma^{-1} \Delta s_\gamma = O(\epstol)$.
We have in addition
\begin{equation}
  G K_\gamma^{-1}\Delta K_\gamma (K_\gamma^{-1} \Delta s_\gamma)  =
  \big[ G U_L \Sigma_L^{-1} \Gamma_L + G U_S \Sigma_S^{-1} \Gamma_S \big] (K_\gamma^{-1} \Delta s_\gamma) \; .
\end{equation}
Using again \eqref{eq:cond:invariantsubpsace}:
$G U_L = G Y + O(\Xi)$ and $G U_S = GZ + O(\Xi) = O(\Xi)$. Hence
$G U_L \Sigma_L^{-1} \Gamma_L = O(1)$ and $G U_S \Sigma_S^{-1} \Gamma_S = O(1)$.
Using \eqref{eq:cond:rhserrorfull}, we have $K_\gamma^{-1} \Delta G^\top = O(\epstol)$,
implying $\Delta G K_\gamma^{-1} \Delta K_\gamma (K_\gamma^{-1} \Delta s_\gamma) = O(\Xi^{-1} \epstol^{2})$.
Assumption~\ref{hyp:cond:wellcond} implies that $\Xi^{-1} \epstol^2 \ll \epstol$.
All in all, the error in the right-hand-side $\Delta \widehat{r}_\gamma$ satisfies:
\begin{equation}
  \label{eq:cond:errorrgamma}
  \Delta \widehat{r}_\gamma = -\Delta \bar{r}_2 + \widehat{G} \widehat{K}_\gamma^{-1} \Delta s_\gamma = O(\epstol) \;.
\end{equation}
The result is remarkable: despite an expression involving the inverse
of the ill-conditioned condensed matrix $K_\gamma$, the error made in $r_\gamma$
is bounded only by the machine precision $\epstol$.

\paragraph{Schur-complement operator.}
The solution of the system~\eqref{eq:kkt:schurcomplhykkt}
involves the Schur complement $S_\gamma = G K_\gamma^{-1} G^\top$.
We show that the Schur complement
has a specific structure that limits the loss of accuracy
in the conjugate gradient algorithm.

\begin{proposition}
  Suppose the current primal-dual iterate $(p, v)$ satisfies Assumption~\ref{hyp:cond:wellcond}.
  In exact arithmetic,
  \begin{equation}
    S_\gamma = GY \, \Sigma_L^{-1} \, Y^\top G^\top + O(\Xi^2) \; .
  \end{equation}
\end{proposition}
\begin{proof}
  Using \eqref{eq:cond:inversecondensed}, we have
  \begin{equation}
    G K_\gamma^{-1} G^\top =
    G U_L \Sigma_L^{-1} U_L^\top G^\top + G U_S \Sigma_S^{-1} U_S^\top G^\top \;.
  \end{equation}
  Using \eqref{eq:cond:invariantsubpsace}, we have $G U_L = GY + O(\Xi)$,
  and $G = O(1)$, implying
  \begin{equation}
    G U_L \Sigma_L^{-1} U_L^\top G^\top = G Y  \Sigma_L^{-1} Y^\top G^\top + O(\Xi^2) \; .
  \end{equation}
  Using again \eqref{eq:cond:invariantsubpsace}, we have $G U_S = GZ + O(\Xi) = O(\Xi)$.
  Hence, $G U_S \Sigma_S^{-1} U_S^\top G^\top = O(\Xi^2)$,
  concluding the proof.
\end{proof}
We adapt the previous proposition to bound the error made when evaluating
$\widehat{S}_\gamma$ in floating-point arithmetic.
\begin{proposition}
  Suppose the current primal-dual iterate $(p, v)$ satisfies Assumption~\ref{hyp:cond:wellcond}.
  In floating-point arithmetic,
  \begin{equation}
    \label{eq:cond:errorSgamma}
    \widehat{S}_\gamma = S_\gamma + O(\epstol) \; .
  \end{equation}
\end{proposition}
\begin{proof}
  We denote $\widehat{G} = G + \Delta G$ (with $G = O(\epstol)$). Then
  \begin{equation}
    \begin{aligned}
      \widehat{S}_\gamma &= \widehat{G} \widehat{K}_\gamma^{-1} \widehat{G}^\top \; , \\
                         &\approx (G + \Delta G)\big(K_\gamma^{-1} - K_\gamma^{-1}\Delta K_\gamma K_\gamma^{-1}\big)(G + \Delta G)^\top \;, \\
                    &\approx S_\gamma - G \big(K_\gamma^{-1}\Delta K_\gamma K_\gamma^{-1} \big)G^\top
                    + K_\gamma^{-1} \Delta G^\top + \Delta G K_\gamma^{-1} \; .
    \end{aligned}
  \end{equation}
  The second line is given by \eqref{eq:cond:invperturbed},
  the third by neglecting the second-order errors.
  Using \eqref{eq:cond:rhserrorfull}, we get $K_\gamma^{-1} \Delta G^\top = O(\epstol)$
  and $\Delta G K_\gamma^{-1} = O(\epstol)$.
  Using \eqref{eq:cond:inversecondensederror}, we have
  \begin{equation*}
    G \big(K_\gamma^{-1}\Delta K_\gamma K_\gamma^{-1} \big)G^\top =
G U_L \Sigma_L^{-1} \Gamma_L \Sigma_L^{-1}U_L^\top G^\top  +
G U_S \Sigma_S^{-1} \Gamma_S \Sigma_S^{-1}U_S^\top  G^\top \;.
  \end{equation*}
  Using \eqref{eq:cond:invariantsubpsace}, we have $G U_S = O(\Xi)$,
  and as $\Sigma_S^{-1} = \Theta(1)$ and $\Gamma_S = O(\Xi^{-1} \epstol)$, we
  get
  $G U_S \Sigma_S^{-1} \Gamma_S \Sigma_S^{-1}U_S^\top  G^\top = O(\Xi \epstol)$.
  Finally, as $\Sigma_L^{-1} = \Theta(\Xi)$ and $G U_L = GY + O(\Xi)$,
  we have
  \begin{equation}
    G U_L \Sigma_L^{-1} \Gamma_L \Sigma_L^{-1}U_L^\top G^\top =
    G Y \Sigma_L^{-1} \Gamma_L \Sigma_L^{-1}Y^\top G^\top + O(\Xi^2 \epstol) \; .
  \end{equation}
  We conclude the proof by using
  $G Y \Sigma_L^{-1} \Gamma_L \Sigma_L^{-1}Y^\top G^\top = O(\Xi \epstol)$.
\end{proof}
The two error bounds \eqref{eq:cond:errorrgamma} and
\eqref{eq:cond:errorSgamma} ensure that we can solve
\eqref{eq:kkt:schurcomplhykkt} using a conjugate gradient
algorithm, as the errors remain limited in floating-point
arithmetic.

\subsubsection{Solution with Lifted KKT system}
The sparse-condensed KKT system has removed the equality
constraints from the optimization problems, simplifying
the solution of the condensed KKT system to
\begin{equation}
  \label{eq:cond:sckktstep}
  K d_x = -r_1 + H^\top (D_s r_4 - r_2) \; .
\end{equation}
Hence, the active Jacobian $A$ reduces to the active inequalities $A = H_{\cactive}$.
Using a similar analysis than in \eqref{eq:cond:boundderivationhykkt},
the error in the right-hand-side is $O(\Xi^{-1} \epstol)$ and is in the
range space of the active Jacobian $A$. Using \eqref{eq:cond:inversecondensed},
we can show that the absolute error on $\widehat{d}_x$ is bounded by $O(\|d_x\| \epstol)
= O(\Xi \epstol)$. That implies the descent direction $\widehat{d}_x$ retains
full relative precision close to optimality.
In other words, we can refine the solution returned by the Cholesky solver accurately using
iterative refinement.

\subsubsection{Summary}
Numerically, the primal-dual step $(\widehat{d}_x, \widehat{d}_y)$
is computed only with an (absolute) precision $\varepsilon_{K}$,
greater than the machine precision $\epstol$ (for HyKKT, $\varepsilon_K$
is the absolute tolerance of the CG algorithm, for LiftedKKT the
absolute tolerance of the iterative refinement algorithm $\varepsilon_{K}$).

The errors $\widehat{d}_x - d_x = O(\varepsilon_K)$ and
$\widehat{d}_y - d_y = O(\varepsilon_K)$ propagate further in $(\widehat{d}_s, \widehat{d}_z)$.
According to \eqref{eq:kkt:condensed}, we have $\widehat{d}_s = - \widehat{r}_4 - \widehat{H} \widehat{d}_x$.
By continuity, $\widehat{H} = H + O(\epstol)$ and $\widehat{r}_4 = r_4 + O(\epstol)$, implying
\begin{equation}
  \widehat{d}_s = d_s + O(\varepsilon_K) \; .
\end{equation}
Eventually, we obtain $\widehat{d}_z = - \widehat{r}_2 - \widehat{D}_s \widehat{d}_s$,
giving the following bounds for the errors in the inactive and active components:
\begin{equation}
  \begin{aligned}
    & \widehat{d}_{z,\cactive} &= -\widehat{r}_{2,\cactive} - \widehat{D}_{s,\cactive} \widehat{d}_{s,\cactive}
    &= d_{z,\cactive} + O(\max\{\epstol, \varepsilon_K \Xi^{-1}\}) \;,\\
                             & \widehat{d}_{z,\cinactive} &= -\widehat{r}_{2,\cinactive} - \widehat{D}_{s,\cinactive} \widehat{d}_{s,\cinactive}
                               &= d_{z,\cinactive} + O(\varepsilon_K \Xi) \; .
  \end{aligned}
\end{equation}
The error arises mostly in the active components $\widehat{d}_{z,\cactive}$.
We want to set $\varepsilon_K \ll \Xi$ to limit the loss of accuracy, but doing
so is not trivial as we are approaching the optimum. For example,
solving the condensed system with a very good tolerance $\varepsilon_K = 10^{-12}$
would only ensure that the absolute error is bounded by $10^{-4}$ if $\Xi = 10^{-8}$
(as it is typical in primal-dual IPM). The impact remains limited if we
have only a few active inequality constraints.


%%% Local Variables:
%%% mode: LaTeX
%%% TeX-master: "../main"
%%% End:

\section{Numerical results}
We implement LiftedKKT and HyKKT on the GPU. For comparison,
we use HSL MA27 and MA57 as a baseline.
First, we detail in \S\ref{sec:num:pprof} the respective performance of the two hybrid solvers
on a large-scale OPF instance. Then, we present in \S\ref{sec:num:opf}
the results reported on the PGLIB OPF benchmark, complemented in \S\ref{sec:num:cops} by
the COPS benchmark.

\subsection{Implementation}
All our implementation uses the Julia language \cite{bezanson-edelman-karpinski-shah-2017}.
We have used our local workstation to generate the results on the CPU, here equipped
with an AMD EPYC 7443 processor (3.2GHz).
For the results on the GPU, we have used an NVIDIA A100 GPU (with CUDA 12.3) on
the Polaris testbed at Argonne National Laboratory
\footnote{\url{https://www.alcf.anl.gov/polaris}}.

\paragraph{IPM solver.}
We have implemented the two condensed-space methods in our nonlinear IPM solver MadNLP~\cite{shin2021graph}.
This implementation utilizes the abstraction {\tt AbstractKKTSystem}
in MadNLP to represent the various formulations of the KKT linear systems.
MadNLP can deport most of the IPM algorithm to the GPU, except for basic IPM operations used for the coordination (e.g., the filter line-search algorithm).
In particular, any operation that involves the manipulation of array entries is performed by GPU kernels without transferring data to host memory.
We refer to \cite{shin2023accelerating} for a detailed description of the GPU implementation in MadNLP.

\paragraph{Evaluation of the nonlinear models.}
We use the ExaModels.jl modeling tool~\cite{shin2023accelerating} to implement all the nonlinear programs utilized in our benchmark.
ExaModels.jl harnesses the sparsity structure and provides custom derivative kernels for repetitive algebraic subexpressions of the constraints and objective functions to compute first and second-order derivatives on the GPU in parallel~\cite{bischof1991exploiting,enzyme2021}.
This approach caters to the SIMD architecture of the GPU by assigning each expression to multiple threads responsible for computing derivatives for different values.

\paragraph{Linear solvers.}
We solve the KKT systems assembled within MadNLP using various sparse linear solvers, chosen based on the KKT formulation (\ref{eq:kkt:condensed}, \ref{eq:kkt:augmented}, \ref{eq:kkt:unreduced}) and the device (CPU, GPU) being utilized. We utilize the following solvers:
\begin{itemize}
  \item {\tt HSL MA27/MA57}: Implement the \lblt factorization on the CPU~\cite{duff1983multifrontal}.
    It solves the augmented KKT system~\eqref{eq:kkt:augmented}.
    This solver serves as the reference when running on the CPU.
  \item {\tt CHOLMOD}: Implements the Cholesky and \ldlt factorizations on the CPU % Sungho: Maybe LDLFactorizations.jl instead, once the results are updated.
    (using the AMD ordering \cite{amestoy-david-duff-2004} by default).
    It factorizes the condensed matrices $K_\gamma$ and $K_\tau$ appearing
    resp. in \eqref{eq:kkt:hykkt} and in \eqref{eq:liftedkkt}.
    This solver is used to assess the performance of the hybrid solvers when running on the CPU.
  \item {\tt cuDSS}: Implement \llt, \ldlt and \lu decompositions on an NVIDIA GPU.
    We use the \ldlt factorization to factorize the ill-conditioned condensed matrices $K_\gamma$ on GPU,
    as \ldlt is more robust than the Cholesky factorization.
  \item {\tt Krylov.jl}: Contains the \CG method
    used in the Golub \& Greif strategy to solve \eqref{eq:kkt:schurcomplhykkt} on both CPU and GPU architectures.
\end{itemize}
CHOLMOD \cite{chen-davis-hager-rajamanickam-2008} is shipped with Julia.
For the HSL linear solvers, we utilize libHSL \cite{fowkes-lister-montoison-orban-2024} with the Julia interface HSL.jl \cite{montoison-orban-hsl-2021}.
HSL MA57 and CHOLMOD are both compiled with OpenBLAS \cite{openblas}, a multithreaded version of BLAS and LAPACK.
The Julia package Krylov.jl~\cite{montoison2023krylov} contains a collection of Krylov methods with a polymorphic implementation that can be used on both CPU and GPU architectures.
% Should we provide the version of the linear solvers?

\subsection{Performance analysis on a large-scale instance}
\label{sec:num:pprof}
We evaluate the performance of each KKT solver on a large-scale OPF instance, taken from
the PGLIB benchmark~\cite{babaeinejadsarookolaee2019power}: {\tt 78484epigrids}.
Our formulation with ExaModels has
a total of 674,562 variables, 661,017 equality constraints and 378,045
inequality constraints.
Our previous work has pointed out that as soon as the OPF model is
evaluated on the GPU using ExaModels, the KKT solver becomes the bottleneck
in the numerical implementation~\cite{shin2023accelerating}.

\subsubsection{Individual performance of the linear solvers}
Subsequently, we evaluate the individual performance of the cuDSS solver when factorizing the matrix $K_{\gamma}$ at the first IPM iteration (here with $\gamma = 10^7$).
We compare the times to perform the symbolic analysis,
the factorization and the triangular solves with those reported in CHOLMOD.

The results are displayed in Table~\ref{tab:linsol:time}.
We benchmark the three decompositions implemented in cuDSS (\llt, \ldlt, \lu), and assess the accuracy of the solution by computing the infinity norm of the residual.
% The accuracy of the solution depends of on more factors than the backsolve.
% The main error comes from the factorization.
We observe that the analysis phase is four times slower for cuDSS compared to CHOLMOD.
Fortunately, this operation needs to be computed only once in IPM, meaning that its cost is amortized if we run many IPM iterations.
The factorization is about twenty times faster in cuDSS, with a time almost independent of the algorithm being used.
The backward and forward sweeps are ten times faster:
the triangular solves are harder to parallelize on a GPU.
% The triangular solves don't parallelize well on a GPU.
In terms of accuracy, the quality of the solution remains on par with CHOLMOD, except for the \ldlt decomposition which lags behind by at least two orders of magnitude.

% \begin{table}[!ht]
%   \centering
%   \resizebox{\textwidth}{!}{
%   \begin{tabular}{|lrrrr|}
%   \hline
%   linear solver & analysis (s) & factorization (s) & backsolve (s) & accuracy \\
%   \hline
%     cholmod     & 0.798 &$6.34\times 10^{-1}$&$5.50\times 10^{-2}$&$3.60\times 10^{-13}$\\
%     cudss-\llt  & 3.50  &$4.77\times 10^{-2}$&$2.08\times 10^{-2}$&$2.64\times 10^{-13}$\\
%     cudss-\lu   & 3.48  &$4.97\times 10^{-2}$&$1.91\times 10^{-2}$&$2.58\times 10^{-13}$\\
%     cudss-\ldlt & 3.49  &$5.80\times 10^{-2}$&$1.88\times 10^{-2}$&$5.44\times 10^{-11}$\\
%   \hline
%   \end{tabular}
%   }
%   \caption{Comparing the performance of cuDSS with CHOLMOD.
%     The matrix $K_\gamma$ is symmetric positive definite, with
%     a size $n = 674,562$. The matrix is extremely sparse, with only $7,342,680$ non-zero entries ($0.002$\%).
%     \label{tab:linsol:time}
%     (A30 GPU)
%   }
% \end{table}

\begin{table}[!ht]
  \centering
  \resizebox{\textwidth}{!}{
  \begin{tabular}{|lrrrr|}
  \hline
  linear solver & analysis (s) & factorization (s) & backsolve (s) & accuracy \\
  \hline
    CHOLMOD     & 1.18  & $8.57 \times 10^{-1}$ & $1.27 \times 10^{-1}$ & $3.60\times 10^{-13}$\\
    cuDSS-\llt   & 4.52  & $3.75 \times 10^{-2}$ & $1.32 \times 10^{-2}$ & $2.64\times 10^{-13}$\\ % Sungho: not Cholesky? it would be better to use consistent term. I'd recommend Cholesky instead of LL^T.
    cuDSS-\lu   & 4.50  & $3.72 \times 10^{-2}$ & $1.49 \times 10^{-2}$ & $2.58\times 10^{-13}$\\
    cuDSS-\ldlt & 4.50  & $4.07 \times 10^{-2}$ & $1.55 \times 10^{-2}$ & $7.62\times 10^{-11}$\\
  \hline
  \end{tabular}
  }
  \caption{Comparing the performance of cuDSS with CHOLMOD.
    The matrix $K_\gamma$ is symmetric positive definite, with
    a size $n = 674,562$. The matrix is extremely sparse, with only $7,342,680$ non-zero entries ($0.002$\%).
    \label{tab:linsol:time}
    (A100 GPU)
  }
\end{table}

\subsubsection{Tuning the Golub \& Greif strategy}
\label{sec:num:tuninghykkt}
In Figure~\ref{fig:hybrid:gamma} we depict the evolution of the number
of \CG iterations and relative accuracy as we increase the parameter $\gamma$
from $10^4$ to $10^8$ in HyKKT.

On the algorithmic side, we observe that the higher the regularization $\gamma$,
the faster the \CG algorithm: we decrease the total number of iterations
spent in \CG by a factor of 10. However, we have to pay a price in term
of accuracy: for $\gamma > 10^8$ the solution returned by the linear solver
is not accurate enough and the IPM algorithm has to proceed to more
primal-dual regularization, leading to an increase in the total number of iterations.

On the numerical side, the table in Figure~\ref{fig:hybrid:gamma} compares
the time spent in the IPM solver on the CPU (using CHOLMOD) and on the GPU
(using the solver {\tt cuDSS}). Overall {\tt cuDSS} is
faster than CHOLMOD, leading to a 4x-8x speed-up in the total IPM solution time.
We note also that the assembly of the condensed matrix $K_\gamma$ parallelizes well
on the GPU, with a reduction in the assembly time from $\approx 8s$ on the CPU to $\approx 0.2s$ on the GPU.

\begin{figure}[!ht]
  \centering
  \resizebox{\textwidth}{!}{
  \begin{tabular}{|r|rrrr >{\bfseries}r|rrrr >{\bfseries}r|}
  \hline
  & \multicolumn{5}{c|}{\bf CHOLMOD (CPU)} & \multicolumn{5}{c|}{\bf cuDSS-\ldlt (CUDA)} \\
  \hline
  $\gamma$ & \# it & cond. (s) & \CG (s) & linsol (s) & IPM (s) & \# it & cond. (s) & \CG (s) & linsol (s) & IPM (s) \\
  \hline
  $10^4$ & 96 & 8.15 & 463.86 & 536.83 & 575.06 & 96 & 0.17 & 113.27 & 114.52 & 124.00 \\
  $10^5$ & 96 & 8.33 & 163.35 & 235.61 & 273.36 & 96 & 0.17 & 53.37 & 54.62 & 64.39 \\
  $10^6$ & 96 & 8.22 & 68.69 & 139.86 & 177.24 & 96 & 0.17 & 14.53 & 15.78 & 25.39 \\
  $10^7$ & 96 & 8.24 & 35.12 & 107.17 & 144.78 & 96 & 0.17 & 7.95 & 9.20 & 18.41 \\
  $10^8$ & 96 & 7.89 & 21.68 & 93.85 & 131.33 & 96 & 0.17 & 5.36 & 6.62 & 15.90 \\
  \hline
  \end{tabular}
  }
  \includegraphics[width=\textwidth]{../figures/hybrid-gamma.pdf}
  \caption{
    Above: Decomposition of IPM solution time across
    (a) condensation time (cond.), (b) \CG time, (c) total time
    spent in the linear solver (linsol.) and (d) total time spent in
    IPM solver (IPM).
    Below: Impact of $\gamma$ on the total number of \CG iterations
    and the norm of the relative residual at each IPM iteration.
    The peak observed in the norm of the relative residual corresponds
    to the primal-dual regularization performed inside the IPM algorithm,
    applied when the matrix $K_\gamma$ is not positive definite.
    \label{fig:hybrid:gamma}
  }
\end{figure}


\subsubsection{Tuning the equality relaxation strategy}
We now analyze the numerical performance of LiftedKKT (\S\ref{sec:kkt:sckkt}).
The method solves the KKT system~\eqref{eq:liftedkkt} using a direct solver.
The parameter $\tau$ used in the equality relaxation~\eqref{eq:problemrelaxation}
is set equal to the IPM tolerance $\varepsilon_{tol}$ (in practice, it does not
make sense to set a parameter $\tau$ below IPM tolerance as the
inequality constraints are satisfied only up to a tolerance $\pm \varepsilon_{tol}$
in IPM).

We compare in Table~\ref{tab:sckkt:performance} the performance obtained by LiftedKKT
as we decrease the IPM tolerance $\varepsilon_{tol}$.
We display both the runtimes on the CPU (using CHOLMOD-\ldlt) and on the GPU (using {\tt cuDSS}-\ldlt).
The slacks associated with the relaxed equality constraints are converging to a value below $2 \tau$,
leading to highly ill-conditioned terms in the diagonal matrices $\Sigma_s$.
As a consequence, the conditioning of the matrix $K_\tau$ in \eqref{eq:liftedkkt} can increase
above $10^{18}$, leading to a nearly singular linear system.
We observe {\tt cuDSS}-\ldlt is more stable: the factorization
succeeds, and the loss of accuracy caused by the ill-conditioning is tamed by the multiple
Richardson iterations that reduces the relative accuracy in the residual down to an acceptable level.
As a result, {\tt cuDSS} can solve
the problem to optimality in $\approx 20s$, a time comparable with HyKKT (see Figure~\ref{fig:hybrid:gamma}).

\begin{table}[!ht]
  \centering
  \resizebox{.7\textwidth}{!}{
  % \begin{tabular}{|l|rr|rr|rr|r|}
  %   \hline
  %   & \multicolumn{2}{c|}{\bf CHOLMOD-\ldlt (CPU)} & \multicolumn{2}{c|}{\bf LDLFactorizations (CPU)} & \multicolumn{2}{c|}{\bf cuDSS-\ldlt (CUDA)}& \\
  %   \hline
  %   $\varepsilon_{tol}$ & \#it & time (s)& \#it & time (s) & \#it & time (s) & accuracy \\
  %   \hline
  %   $10^{-4}$& 115 & 268.2  &220& 358.8& 114 & 19.9& $1.2 \times 10^{-2}$\\
  %   $10^{-5}$ & 210 & 777.8 & 120&  683.1& 113 & 30.4&$1.2 \times 10^{-3}$ \\
  %   $10^{-6}$ & 102 & 337.5 & 109&  328.7& 109 & 25.0&$1.2 \times 10^{-4}$  \\
  %   $10^{-7}$ &108 & 352.9 & 108&  272.9& 104 & 20.1&$1.2 \times 10^{-5}$ \\
  %   $10^{-8}$ & - &  - &- &  - & 105 & 20.3&$1.2 \times 10^{-6}$  \\
  %   \hline
  % \end{tabular}
  \begin{tabular}{|l|rr|rr|rr|}
    \hline
    & \multicolumn{2}{c|}{\bf CHOLMOD-\ldlt (CPU)} & \multicolumn{2}{c|}{\bf cuDSS-\ldlt (CUDA)}& \\
    \hline
    $\varepsilon_{tol}$ & \#it & time (s)&  \#it & time (s) & accuracy \\
    \hline
    $10^{-4}$ & 115 & 268.2 & 114 & 19.9 & $1.2 \times 10^{-2}$ \\
    $10^{-5}$ & 210 & 777.8 & 113 & 30.4 & $1.2 \times 10^{-3}$ \\
    $10^{-6}$ & 102 & 337.5 & 109 & 25.0 & $1.2 \times 10^{-4}$ \\
    $10^{-7}$ & 108 & 352.9 & 104 & 20.1 & $1.2 \times 10^{-5}$ \\
    $10^{-8}$ & -   & -     & 105 & 20.3 & $1.2 \times 10^{-6}$ \\
    \hline
  \end{tabular}
  }
  \label{tab:sckkt:performance}
  \caption{Performance of the equality-relaxation
    strategy as we decrease the IPM tolerance $\varepsilon_{tol}$.
    The table displays the wall time on the CPU (using CHOLMOD-\ldlt)
    and on the GPU (using cuDSS-\ldlt).
  }
\end{table}

\subsubsection{Breakdown of the time spent in one IPM iteration}
We decompose the time spent in a single
IPM iteration for all the available KKT solvers (HSL MA27, LiftedKKT, and HyKKT).
When solving the KKT system, the time can be decomposed into: (1) assembling the
KKT system, (2) factorizing the KKT system, and (3) computing the descent direction with triangular solves.
As depicted in Figure~\ref{fig:timebreakdown}, we observe
that constructing the KKT system represents only a fraction of the computation time, compared
to the factorization and the triangular solves. Using {\tt cuDSS}-\ldlt, we observe speedups of
30x and 15x in the factorization compared to MA27 and CHOLMOD running on the CPU.
Once the KKT system is factorized, computing the descent direction with LiftedKKT is faster than with HyKKT
(0.04s compared to 0.13s) as HyKKT has to run a \CG algorithm to solve the Schur complement
system~\eqref{eq:kkt:schurcomplhykkt}.

\begin{figure}[!ht]
  \centering
  \resizebox{.8\textwidth}{!}{
    \begin{tabular}{|l|rrrr|}
      \hline
       & build (s) & factorize (s) & backsolve (s) & accuracy \\
       \hline
      HSL MA27       & $3.15\times 10^{-2}$&$1.22 \times 10^{-0} $&$3.58\times 10^{-1}$&$5.52\times 10^{-7}$\\
      LiftedKKT (CPU)  & $8.71\times 10^{-2}$&$6.08\times 10^{-1}$&$2.32\times 10^{-1}$&$3.73\times 10^{-9}$\\
      HyKKT (CPU)  & $7.97\times 10^{-2}$&$6.02\times 10^{-1}$&$7.30\times 10^{-1}$&$3.38\times 10^{-3}$\\
      LiftedKKT (CUDA) & $2.09\times 10^{-3}$&$4.37\times 10^{-2}$&$3.53\times 10^{-2}$&$4.86\times 10^{-9}$\\
      HyKKT (CUDA) & $1.86\times 10^{-3}$&$3.38\times 10^{-2}$&$1.35\times 10^{-1}$&$3.91\times 10^{-3}$\\
      \hline
    \end{tabular}
  }
  \includegraphics[width=.7\textwidth]{../figures/breakdown.pdf}
  \caption{Breakdown of the time spent in one IPM iteration
    for different linear solvers, when solving {\tt 78484epigrids} (A30 GPU)
  \label{fig:timebreakdown}}
\end{figure}



\subsection{Benchmark on OPF instances}
\label{sec:num:opf}
We run a benchmark on difficult OPF instances taken
from the PGLIB benchmark~\cite{babaeinejadsarookolaee2019power}.
We compare our two condensed-space methods with HSL MA27 running
on the CPU. The results are displayed in Table~\ref{tab:opf:benchmark},
for an IPM tolerance set to $10^{-6}$.
Regarding HyKKT, we set $\gamma = 10^7$ following the analysis in \S\ref{sec:num:tuninghykkt}.
The table displays the time spent in the initialization, the time spent in the linear solver and the total
solving time.
We complement the table with a Dolan \& Moré performance profile~\cite{dolan2002benchmarking} displayed
in Figure~\ref{fig:opf:pprof}.

Overall, the performance of HSL MA27 on the CPU is consistent with what was reported
in \cite{babaeinejadsarookolaee2019power}: We observe that HSL MA57 is slower
than HSL MA27, as the OPF instances are super-sparse.
Hence, the block elimination algorithm implemented in HSL MA57 is not beneficial there
\footnote{Personal communication with Iain Duff.}.

On the GPU, LiftedKKT+cuDSS is faster than HyKKT+cuDSS on small and medium instances: indeed, the algorithm
does not have to run a \CG algorithm at each IPM iteration, limiting the number
of triangular solves performed at each iteration.
Both LiftedKKT+cuDSS and HyKKT+cuDSS are significantly faster than HSL MA27.
HyKKT+cuDSS is slower when solving {\tt 8387\_pegase}, as on this particular instance
the parameter $\gamma$ is not set high enough to reduce the
total number of \CG iterations, leading to a 4x slowdown compared to LiftedKKT+cuDSS.
Nevertheless, the performance of HyKKT+cuDSS is better on the largest instances,
with almost an 8x speed-up compared to the reference HSL MA27.

The benchmark presented in Table~\ref{tab:opf:benchmark} has been generated using a
NVIDIA A100 GPUs (current selling price: \$10k). We have also compared the performance
with cheaper GPUs: a NVIDIA A1000 (a laptop-based GPU, 4GB memory, price: \$150) and a NVIDIA A30
(24GB memory, price: \$5k).
As a comparison, the selling price of the AMD EPYC
7443 processor we used for the benchmark on the CPU is \$1.2k. The results are
displayed in Figure~\ref{fig:gpubench}. We observe that the performance of the A30
and the A100 are relatively similar. The cheaper A1000 GPU is already faster
than the GPU, but is not able to solve the largest instance as it is running out of memory.

\begin{table}[!ht]
  \centering
  \resizebox{\textwidth}{!}{
    \begin{tabular}{|l|rrr >{\bfseries}r|rrr >{\bfseries}r|rrr >{\bfseries}r|}
      \hline
      & \multicolumn{4}{c|}{\bf HSL MA27} &
      \multicolumn{4}{c|}{\bf LiftedKKT+cuDSS} &
      \multicolumn{4}{c|}{\bf HyKKT+cuDSS} \\
      \hline
      Case & it & init & lin & total & it & init & lin & total & it & init & lin & total \\
      \hline
      89\_pegase & 32 & 0.00 & 0.02 & 0.03 & 29 & 0.03 & 0.12 & 0.24 & 32 & 0.03 & 0.07 & 0.22 \\
      179\_goc & 45 & 0.00 & 0.03 & 0.05 & 39 & 0.03 & 0.19 & 0.35 & 45 & 0.03 & 0.07 & 0.25 \\
      500\_goc & 39 & 0.01 & 0.10 & 0.14 & 39 & 0.05 & 0.09 & 0.26 & 39 & 0.05 & 0.07 & 0.27 \\
      793\_goc & 35 & 0.01 & 0.12 & 0.18 & 57 & 0.06 & 0.27 & 0.52 & 35 & 0.05 & 0.10 & 0.30 \\
      1354\_pegase & 49 & 0.02 & 0.35 & 0.52 & 96 & 0.12 & 0.69 & 1.22 & 49 & 0.12 & 0.17 & 0.50 \\
      \hline
      2000\_goc & 42 & 0.03 & 0.66 & 0.93 & 46 & 0.15 & 0.30 & 0.66 & 42 & 0.16 & 0.14 & 0.50 \\
      2312\_goc & 43 & 0.02 & 0.59 & 0.82 & 45 & 0.14 & 0.32 & 0.68 & 43 & 0.14 & 0.21 & 0.56 \\
      2742\_goc & 125 & 0.04 & 3.76 & 7.31 & 157 & 0.20 & 1.93 & 15.49 & - & - & - & - \\
      2869\_pegase & 55 & 0.04 & 1.09 & 1.52 & 57 & 0.20 & 0.30 & 0.80 & 55 & 0.21 & 0.26 & 0.73 \\
      3022\_goc & 55 & 0.03 & 0.98 & 1.39 & 48 & 0.18 & 0.23 & 0.66 & 55 & 0.18 & 0.23 & 0.68 \\
      \hline
      3970\_goc & 48 & 0.05 & 1.95 & 2.53 & 47 & 0.26 & 0.37 & 0.87 & 48 & 0.27 & 0.24 & 0.80 \\
      4020\_goc & 59 & 0.06 & 3.90 & 4.60 & 123 & 0.28 & 1.75 & 3.15 & 59 & 0.29 & 0.41 & 1.08 \\
      4601\_goc & 71 & 0.09 & 3.09 & 4.16 & 67 & 0.27 & 0.51 & 1.17 & 71 & 0.28 & 0.39 & 1.12 \\
      4619\_goc & 49 & 0.07 & 3.21 & 3.91 & 49 & 0.34 & 0.59 & 1.25 & 49 & 0.33 & 0.31 & 0.95 \\
      4837\_goc & 59 & 0.08 & 2.49 & 3.33 & 59 & 0.29 & 0.58 & 1.31 & 59 & 0.29 & 0.35 & 0.98 \\
      \hline
      4917\_goc & 63 & 0.07 & 1.97 & 2.72 & 55 & 0.26 & 0.55 & 1.18 & 63 & 0.26 & 0.34 & 0.94 \\
      5658\_epigrids & 51 & 0.31 & 2.80 & 3.86 & 58 & 0.35 & 0.66 & 1.51 & 51 & 0.35 & 0.35 & 1.03 \\
      7336\_epigrids & 50 & 0.13 & 3.60 & 4.91 & 56 & 0.45 & 0.95 & 1.89 & 50 & 0.43 & 0.35 & 1.13 \\
      8387\_pegase & 74 & 0.14 & 5.31 & 7.62 & 82 & 0.59 & 0.79 & 2.30 & 75 & 0.58 & 7.66 & 8.84 \\
      9241\_pegase & 74 & 0.15 & 6.11 & 8.60 & 101 & 0.63 & 0.88 & 2.76 & 71 & 0.63 & 0.99 & 2.24 \\
      \hline
      9591\_goc & 67 & 0.20 & 11.14 & 13.37 & 98 & 0.63 & 2.67 & 4.58 & 67 & 0.62 & 0.74 & 1.96 \\
      10000\_goc & 82 & 0.15 & 6.00 & 8.16 & 64 & 0.49 & 0.81 & 1.83 & 82 & 0.49 & 0.75 & 1.82 \\
      10192\_epigrids & 54 & 0.41 & 7.79 & 10.08 & 57 & 0.67 & 1.14 & 2.40 & 54 & 0.67 & 0.66 & 1.81 \\
      10480\_goc & 71 & 0.24 & 12.04 & 14.74 & 67 & 0.75 & 0.99 & 2.72 & 71 & 0.74 & 1.09 & 2.50 \\
      13659\_pegase & 63 & 0.45 & 7.21 & 10.14 & 75 & 0.83 & 1.05 & 2.96 & 62 & 0.84 & 0.93 & 2.47 \\
      \hline
      19402\_goc & 69 & 0.63 & 31.71 & 36.92 & 73 & 1.42 & 2.28 & 5.38 & 69 & 1.44 & 1.93 & 4.31 \\
      20758\_epigrids & 51 & 0.63 & 14.27 & 18.21 & 53 & 1.34 & 1.05 & 3.57 & 51 & 1.35 & 1.55 & 3.51 \\
      30000\_goc & 183 & 0.65 & 63.02 & 75.95 & - & - & - & - & 225 & 1.22 & 5.59 & 10.27 \\
      78484\_epigrids & 102 & 2.57 & 179.29 & 207.79 & 101 & 5.94 & 5.62 & 18.03 & 104 & 6.29 & 9.01 & 18.90 \\
      \hline
    \end{tabular}
  }
  \caption{OPF benchmark, solved with a tolerance {\tt tol=1e-6}. (A100 GPU) \label{tab:opf:benchmark}}
\end{table}

\begin{figure}[!ht]
  \centering
  \includegraphics[width=.6\textwidth]{../figures/pprof-cuda.pdf}
  \caption{Performance profile for the PGLIB OPF benchmark, solved
    with a tolerance {\tt tol=1e-6}.
  \label{fig:opf:pprof}}
\end{figure}

\begin{figure}[!ht]
  \centering
  \includegraphics[width=.6\textwidth]{../figures/benchmark_gpus.pdf}
  \caption{Comparing the performance obtained with various GPUs
    on three different OPF instances.
  \label{fig:gpubench}}
\end{figure}


\subsection{Benchmark on COPS instances}
\label{sec:num:cops}
We have observed in the previous section that both LiftedKKT
and HyKKT outperforms HSL MA27 when running on the GPU.
However, the OPF instances are specific nonlinear instances.
For that reason, we complement our analysis by looking
at the performance of LiftedKKT and HyKKT on the COPS benchmark,
which gathers generic nonlinear programs~\cite{dolan2004benchmarking}.
We look at the performance we get on the particular COPS instances used in
the Mittelmann benchmark, used to benchmark nonlinear optimization
solvers~\cite{mittelmann2002benchmark}.
To illustrate the heterogeneity of the COPS instances,
we display in Figure~\ref{fig:cops:nnz} the sparsity pattern of the
condensed matrices $K_\gamma$ \eqref{eq:kkt:hykkt} for one OPF instance and for multiple
COPS instances. We observe that some instances ({\tt bearing}) have a sparsity pattern
similar to the OPF instance on the left, whereas some are fully dense ({\tt elec}).
On the opposite, the optimal control instances ({\tt marine}, {\tt steering}) are
highly sparse and can be reordered efficiently using AMD ordering~\cite{amestoy-david-duff-2004}.

The results of the COPS benchmark are displayed in Table~\ref{tab:cops:benchmark}.
HSL MA57 gives better results than HSL MA27 for the COPS benchmark, and
for that reason we have decided to use HSL MA57 as the reference.  As expected,
the results are different than on the OPF benchmark.
We observe that LiftedKKT+cuDSS and HyKKT+cuDSS outperform HSL MA57 on the dense instance {\tt elec}
(20x speed-up) and {\tt bearing}  --- an instance whose sparsity pattern
is similar to the OPF. In the other instances, LiftedKKT+cuDSS and HyKKT+cuDSS
on par with HSL MA57 and sometimes even slightly slower ({\tt rocket} and {\tt pinene}).


\begin{figure}[!ht]
  \centering
  \includegraphics[width=.9\textwidth]{../figures/sparsity_pattern.png}
  \caption{Sparsity patterns for one OPF instance and for various
    COPS problems. The first row displays the sparsity pattern
    of $K_\gamma$, after AMD reordering. The second row displays
    the sparsity pattern of the triangular factor computed by CHOLMOD.
    \label{fig:cops:nnz}
  }
\end{figure}


\begin{table}[!ht]
  \centering
  \resizebox{\textwidth}{!}{
    \begin{tabular}{|l|rr|rrr >{\bfseries}r|rrr >{\bfseries}r|rrr >{\bfseries}r|}
      \hline
  & &
  & \multicolumn{4}{c|}{\bf HSL MA57} &
      \multicolumn{4}{c|}{\bf LiftedKKT+cuDSS} &
      \multicolumn{4}{c|}{\bf HyKKT+cuDSS} \\
      \hline
      & $n$ & $m$ & it & init & lin & total & it & init & lin & total & it & init & lin & total \\
      \hline
      bearing\_400 & 162k & 2k & 17 & 0.14 & 3.42 & 4.10 & 14 & 0.85 & 0.07 & 1.13 & 14 & 0.78 & 0.76 & 1.74 \\
      camshape\_6400 & 6k & 19k & 38 & 0.02 & 0.18 & 0.29 & 35 & 0.05 & 0.03 & 0.19 & 38 & 0.05 & 0.04 & 0.23 \\
      elec\_400 & 1k & 0.4k & 185 & 0.54 & 24.64 & 33.02 & 273 & 0.46 & 0.97 & 20.01 & 128 & 0.48 & 0.75 & 4.16 \\
      gasoil\_3200 & 83k & 83k & 37 & 0.36 & 4.81 & 5.81 & 21 & 0.54 & 0.24 & 1.40 & 20 & 0.59 & 0.21 & 1.35 \\
      marine\_1600 & 51k & 51k & 13 & 0.05 & 0.41 & 0.50 & 33 & 0.38 & 0.58 & 1.29 & 13 & 0.37 & 0.12 & 0.62 \\
      pinene\_3200 & 160k & 160k & 12 & 0.11 & 1.32 & 1.60 & 21 & 0.87 & 0.16 & 1.52 & 11 & 0.90 & 0.84 & 2.02 \\
      robot\_1600 & 14k & 10k & 34 & 0.04 & 0.33 & 0.45 & 35 & 0.20 & 0.07 & 0.76 & 34 & 0.21 & 0.08 & 0.80 \\
      rocket\_12800 & 51k & 38k & 23 & 0.12 & 1.73 & 2.16 & 37 & 0.74 & 0.06 & 2.49 & 24 & 0.25 & 1.70 & 3.12 \\
      steering\_12800 & 64k & 51k & 19 & 0.25 & 1.49 & 1.93 & 18 & 0.44 & 0.06 & 1.64 & 18 & 0.46 & 0.07 & 1.83 \\
      \hline
      bearing\_800 & 643k & 3k & 13 & 0.94 & 14.59 & 16.86 & 14 & 3.31 & 0.18 & 4.10 & 12 & 3.32 & 1.98 & 5.86 \\
      camshape\_12800 & 13k & 38k & 34 & 0.02 & 0.34 & 0.54 & 33 & 0.05 & 0.02 & 0.16 & 34 & 0.06 & 0.03 & 0.19 \\
      elec\_800 & 2k & 0.8k & 354 & 2.36 & 337.41 & 409.57 & 298 & 2.11 & 2.58 & 24.38 & 184 & 1.81 & 2.40 & 16.33 \\
      gasoil\_12800 & 333k & 333k & 20 & 1.78 & 11.15 & 13.65 & 18 & 2.11 & 0.98 & 5.50 & 22 & 2.99 & 1.21 & 6.47 \\
      marine\_12800 & 410k & 410k & 11 & 0.36 & 3.51 & 4.46 & 146 & 2.80 & 25.04 & 39.24 & 11 & 2.89 & 0.63 & 4.03 \\
      pinene\_12800 & 640k & 640k & 10 & 0.48 & 7.15 & 8.45 & 21 & 4.50 & 0.99 & 7.44 & 11 & 4.65 & 3.54 & 9.25 \\
      robot\_12800 & 115k & 77k & 35 & 0.54 & 4.63 & 5.91 & 33 & 1.13 & 0.30 & 4.29 & 35 & 1.15 & 0.27 & 4.58 \\
      rocket\_51200 & 205k & 154k & 31 & 1.21 & 6.24 & 9.51 & 37 & 0.83 & 0.17 & 8.49 & 30 & 0.87 & 2.67 & 10.11 \\
      steering\_51200 & 256k & 205k & 27 & 1.40 & 9.74 & 13.00 & 15 & 1.82 & 0.19 & 5.41 & 28 & 1.88 & 0.56 & 11.31 \\
      \hline
    \end{tabular}
  }
  \caption{COPS benchmark , solved with a tolerance {\tt tol=1e-6}\label{tab:cops:benchmark} (A100 GPU)}
\end{table}


%%% Local Variables:
%%% mode: latex
%%% TeX-master: "../main"
%%% End:


\section{Conclusion}
This article moves one step further in the solution of generic nonlinear
programs on GPU architectures. We have compared two approaches
to solve the KKT systems arising at each interior-point iteration, both
based on a condensation procedure. Despite the forming of a ill-conditioned
matrix, our theoretical analysis show that the loss of accuracy is benign
in floating point arithmetic, thanks to the specific properties of IPM.
Our numerical results show that both methods are competitive to solve large-scale
nonlinear programs. Compared to the state-of-the-art linear solvers in HSL, we obtain 10x speed-up on large-scale OPF instances
and quasi-dense instances ({\tt elec}). The results are more mitigated
on the various instances of the COPS benchmark, but the performance remains
always competitive with HSL.

In the future, we are planning to robustify the performance of
the two condensed KKT methods, in particular, to stabilize the convergence
for small tolerance (below $10^{-8}$). Notably, the sparse Cholesky solver
can be adapted further to the specific needs of the IPM~\cite{wright1999modified}.
Improving the two methods on the GPU would allow to solve large-scale
problems intractable on the classical CPU architecture --- such as the multiperiod
and security-constrained OPF problems --- especially when used in conjunction
with a Schur-complement approach.


\bibliographystyle{siam}
\bibliography{biblio.bib}

\end{document}
